% !TEX root = ../fig.tex

\begin{lemma}[Algorithmic Weakening]
    \label{lemma:AlgorithmicWeakening}
    If \(\G \AVS \mathcal{J}@A\) and \(\G\) is a sub-sequence of \(\D\), then \(\D \AVS J@A\).
\end{lemma}

\begin{proof}
    \begin{rneqncase}{\TAVar}{
        \G \AVS x:\tau@A
        \text{ is derived from }
        x:\tau@A \in \G
        }
        Because \( \G \) is a sub-sequence of \( \D \), \( x:\tau@A \in \D \).
        Therefore, we can use \TAVar.
    \end{rneqncase}
    \begin{rneqncase}{\TAAbs}{
            \G\AVS(\lambda (x:\sigma).M):(\Pi (x:\sigma).\tau)@A
            \text{ is derived from }
            \G\AVS \sigma::*@A \text{ and }
            \G,x:\sigma@A\AVS M:\tau@A.
        }
        By the induction hypothesis, \( \D \AVS \sigma::*@A \) and \(
        \D,x:\sigma@A\AVS M:\tau@A. \). Then, we can get it from \TAAbs.
    \end{rneqncase}
    \begin{rneqncase}{\QAANF}{
            \G \AVS M \E N@A
            \text{ is derived from } \\
            \ANF(M) \E_\alpha \ANF(N),
            \G \AVS M : \tau @ A \text{ and }
            \G \AVS N : \tau' @ A.
        }
        From the induction hypothesis, \( \D \AVS M : \tau @ A \), \( \D \AVS N
        : \tau' @ A \) and \( \D \AVS \tau \E \tau' @ A \). Then, we can use \QAANF.
    \end{rneqncase}
\end{proof}

\begin{lemma}[Strong Normalization of Algorithmic Reduction]
    \label{lemma:StrongNormalizationofAlgorithmicReduction}
    If \( \G \AVS M :\tau @ A\), there is no infinite sequence of \( \RA \) reductions from \( M \).
\end{lemma}

\begin{proof}
    Proved by using \( \natural \) function.
\end{proof}

\begin{lemma}[Confluence of Algorithmic Reduction]
    \label{lemma:ConfluenceofAlgorithmicReduction}
    If \( \G \AVS M :\tau @ A\), \( M \RA^* M' \) and \(M \RA^* M''\)
    then there is a \( M''' \) such that \( M' \RA^* M''' \) and \( M'' \RA^* M''' \).
\end{lemma}

\begin{proof}
    It can be proved just same as the normal reduction.
\end{proof}

\begin{lemma}[Uniqueness of Algorithmic Reduction Normal Form]
    \label{lemma:UniquenessOfANF}
    If \( \G \AVS M : \tau @ A \) then there is the unique normal form of \( M \), \( \ANF(M) \).
\end{lemma}

\begin{proof}
    It is obvious from Lemma \ref{lemma:StrongNormalizationofAlgorithmicReduction} and \ref{lemma:ConfluenceofAlgorithmicReduction}.
\end{proof}

\begin{lemma}[Algorithmic Reduction and Term Substitution]
    \label{lemma:AlgorithmicReductionAndTermSubstitution}
    If \( \G, z:\xi@B, \D \AVS M : \tau @ A \) and \( M \RA M' \) then
    \( \ANF(M[z\mapsto P]) \E_\alpha \ANF(M'[z\mapsto P]) \).
\end{lemma}

\begin{proof}
    Prove by induction on the derivation of \( M \RA M' \).
    \begin{rneqncase}{\textsc{A-\%}}{\%M \RA M \text{ and } \text{FV}(M)=\emptyset}
        From \( \text{FV}(M)=\emptyset \), \( z \notin \text{FV}(M) \) and \( z \notin \text{FV}(M') \).
        Then \( \ANF(M) \E_\alpha \ANF(M') \) is obvious.
    \end{rneqncase}

    \begin{rneqncase}{\textsc{A-$\beta$}}{(\lambda x:\tau.M)\ N \RA M[x\mapsto N]}
        \begin{align*}
            & \ANF(((\lambda x:\tau.M)\ N)[z \mapsto P]) \\
            & \E_\alpha \ANF((\lambda x:\tau.M[z \mapsto P])\ N[z \mapsto P]) \\
            & \E_\alpha \ANF(M[z \mapsto P][x \mapsto N[z \mapsto P]]) & (\text{From Lemma } \ref{lemma:UniquenessOfANF}) \\
            & \E_\alpha \ANF(M[x \mapsto N][z \mapsto P]) \\
            & \E_\alpha \ANF(M'[z \mapsto P])
        \end{align*}
    \end{rneqncase}
\end{proof}

\begin{lemma}[Algorithmic Reduction and Stage Substitution]
    \label{lemma:AlgorithmicReductionAndStageSubstitution}
    If \( \G \AVS M : \tau @ A \) and \( M \RA M' \) then
    \( \ANF(M[\alpha \mapsto B]) \E_\alpha \ANF(M'[\alpha \mapsto B]) \).
\end{lemma}

\begin{proof}
    Prove by induction on the derivation of \( M \RA M' \).
    \begin{rneqncase}{\textsc{A-$\Lambda$}}{(\Lambda \beta.M)\ C \RA M[\beta \mapsto C]}
        \begin{align*}
            & \ANF(((\Lambda \beta.M)\ C)[\alpha \mapsto B]) \\
            & \E_\alpha \ANF((\Lambda \beta.M[\alpha \mapsto B])\ C[\alpha \mapsto B]) \\
            & \E_\alpha \ANF((M[\alpha \mapsto B])[\beta \mapsto C[\alpha \mapsto B]]) \\
            & \E_\alpha \ANF((M[\beta \mapsto C])[\alpha \mapsto B])
        \end{align*}
    \end{rneqncase}
\end{proof}

\begin{lemma}[Algorithmic Normal Form and Term Substitution]
    \label{lemma:AlgorithmicNomalFormAndTermSubstitution}
    If \\ \( \G, z:\xi@B, \D \AVS M : \tau @ A \) then
    \( \ANF(M[z\mapsto P]) \E_\alpha \ANF(\ANF(M)[z\mapsto P]) \)
\end{lemma}

\begin{proof}
    Prove by induction on the length \( n \) of reduction from \( M \) to \( \ANF(M) \).
    \begin{rneqncase}{$n$ = 0}{M = \ANF(M) }
        It is obvious.
    \end{rneqncase}
    \begin{rneqncase}{$n$ = 1}{M \RA M' \text{ and } \ANF(M) = M' }
        From Lemma \ref{lemma:AlgorithmicReductionAndTermSubstitution}.
    \end{rneqncase}
    \begin{rneqncase}{$n$ > 1}{M \RA^* M' \text{ and } M' \RA^* M'' \text{ and } \ANF(M) = M'' }
        From the induction hypothesis, \( \ANF(M[z\mapsto P]) \E_\alpha \ANF(M'[z\mapsto P]) \).
        From Lemma \ref{lemma:AlgorithmicReductionAndTermSubstitution}, \( \ANF(M'[z\mapsto P]) \E_\alpha \ANF(M''[z\mapsto P]) \).
        Then from the transitivity of \( \E_\alpha \), \( \ANF(M[z\mapsto P]) \E_\alpha \ANF(\ANF(M)[z\mapsto P]) \).
    \end{rneqncase}
\end{proof}

\begin{lemma}[Algorithmic Normal Form and Stage Substitution]
    \label{lemma:AlgorithmicNomalFormAndStageSubstitution}
    If \\ \( \G \AVS M : \tau @ A \) then
    \( \ANF(M[\alpha \mapsto B]) \E_\alpha \ANF(\ANF(M)[\alpha \mapsto A]) \)
\end{lemma}

\begin{proof}
    Same as Lemma \ref{lemma:AlgorithmicNomalFormAndTermSubstitution} using
    Lemma \ref{lemma:AlgorithmicReductionAndStageSubstitution} instead of Lemma
    \ref{lemma:AlgorithmicReductionAndTermSubstitution}.
\end{proof}

\begin{lemma}[Term Substitution Lemma of Algorithmic Judgement]
    \label{lemma:TermSubstitutionLemmaOfAlgorithmicJudgement}
    If $\G, z:\xi@B, \D \AVS \mathcal{J}$ and $\G\AVS P:\xi @B$ then $\G, \D[z \mapsto P] \AVS \mathcal{J}[z \mapsto P]$.
\end{lemma}

\begin{proof}
    Prove by induction on the derivation of \( \G, z:\xi@B, \D \AVS \mathcal{J}
    \). Most cases are the same with Lemma \ref{lemma:TermSubstitution}. I show
    only different cases.

    \begin{rneqncase}{\QAANF}{
            \G, z:\xi @ B, \D \AVS M \E N @ A
            \text{ is derived from } \\
            \ANF(M) \E_\alpha \ANF(N),
            \G, z:\xi @ B, \D \AVS M : \tau @ A, \text{ and }
            \G, z:\xi @ B, \D \AVS N : \tau' @ A.
        }
    \end{rneqncase}
    \( \ANF(\ANF(M)[z \mapsto P]) \E_\alpha \ANF(\ANF(N)[z \mapsto P]) \) is
    obvious from \( \ANF(M) \E_\alpha \ANF(N) \).  From Lemma
    \ref{lemma:AlgorithmicNomalFormAndTermSubstitution}, \( \ANF(M[z \mapsto
    P]) \E_\alpha \ANF(N[z \mapsto P]) \). Using the induction hypothesis on
    the premises of \QAANF, get \( \G, \D[z \mapsto P] \AVS M : \tau @ A \) and
    \( \G, \D[z \mapsto P] \AVS N : \tau' @ A \). Then \( \G, \D[z \mapsto P]
    \AVS M[z \mapsto P] \E N[z \mapsto P] @ A \) from \QAANF.
\end{proof}

\begin{lemma}[Stage Substitution Lemma of Algorithmic Judgement]
    \label{lemma:StageSubstitutionLemmaofAlgorithmicJudgement}
    If $\G \AVS \mathcal{J}$ then $\G[\alpha \mapsto B] \AVS \mathcal{J}[\alpha \mapsto B]$.
\end{lemma}

\begin{proof}
    Same as Lemma \ref{lemma:TermSubstitutionLemmaOfAlgorithmicJudgement} using
    Lemma \ref{lemma:AlgorithmicNomalFormAndStageSubstitution} instead of Lemma
    \ref{lemma:AlgorithmicNomalFormAndTermSubstitution}.
\end{proof}

% \begin{lemma}[Algorithmic Type Preservation of Algorithmic Reduction]
%     \label{lemma:AlgorithmicTypePreservationofAlgorithmicReduction}
%     If \( \G \AVS M:\tau @A \) and \( M \RA M' \) then \( \G \AVS M' : \tau @ A \).
% \end{lemma}
%
% \begin{proof}
%     Similar to Theorem \ref{theorem:TypePreservation}. In the case of
%     \textsc{A-$\beta$} and \textsc{A-$\Lambda$} we can prove it by Lemma
%     \ref{lemma:TermSubstitutionLemmaOfAlgorithmicJudgement} and
%     \ref{lemma:StageSubstitutionLemmaofAlgorithmicJudgement}.
% \end{proof}

\begin{lemma}[Agreement of Algorithmic Typing]
    Following statements are valid simultaneously.
    \label{lemma:AgreementofAlgorithmicTyping}
    \begin{itemize}
        \item If \( \G \AVS \tau :: K @ A \) then \( \G \AVS K \iskind@ A \).
        \item If \( \G \AVS M : \tau @ A \) then \( \G \AVS \tau :: * @ A \).
    \end{itemize}
\end{lemma}

\begin{proof}
    \begin{rneqncase}{\TACsp}{
        \G\AVS \%_\alpha M:\tau@{A\alpha}
        \text{ is derived from }
        \G\AVS M:\tau@A \text{ and } \G\AVS \tau :: * @A\alpha.
        }
        \( \G\AVS \tau :: * @A\alpha \) is obvious.
    \end{rneqncase}
    \begin{rneqncase}{\TAConst}{
        \G \AVS c:\tau@A
        \text{ is derived from }
        c:\tau \in \Sigma
        }
        From the well-formedness of \( \Sigma \), \( \AVS \tau :: * @
        \varepsilon \). Then, we can derive \( \AVS \tau :: * @ A \) from
        Lemma~\ref{lemma:AlgorithmicWeakening} because there is no free
        variables or type variables in \( \tau \) and all components of \( \tau
        \) is included in \( \Sigma \).
    \end{rneqncase}
    \begin{rneqncase}{\KATConst}{
            \G \AVS X::K@A
            \text{ is derived from }
            X::K \in \Sigma.
        }
        From the well-formedness of \( \Sigma \), \( \AVS K \iskind @
        \varepsilon \).  Then, we can derive \( \AVS K \iskind @ A \) from
        Lemma~\ref{lemma:AlgorithmicWeakening} because there is no free
        variables or type variables in \( K \) and all components of \( \tau \)
        is included in \( \Sigma \).
    \end{rneqncase}
\end{proof}

\begin{lemma}[Symmetricity in Algorithmic Equivalence]
    The algorithmic equivalence relationship is symmetrical.
    \label{lemma:SymmetricityinAlgorithmicEquivalence}
    \begin{itemize}
        \item If \( \G \AVS K \E K' @ A \) then \( \G \AVS K' \E K @ A \).
        \item If \( \G \AVS \tau \E \tau' @ A \) then \( \G \AVS \tau' \E \tau @ A \).
        \item If \( \G \AVS M \E M' @ A \) then \( \G \AVS M' \E M @ A \).
    \end{itemize}
\end{lemma}

\begin{proof}
    It is obvious because all algorithmic equivalence rules are symmetrical.
\end{proof}

\begin{lemma}[Reflexivity in Algorithmic Equivalence]
    The algorithmic equivalence relationship is reflexive.
    \label{lemma:ReflexivityinAlgorithmicEquivalence}
    \begin{itemize}
        \item \( \G \AVS K \E K @ A \).
        \item \( \G \AVS \tau \E \tau @ A \).
        \item \( \G \AVS M \E M @ A \).
    \end{itemize}
\end{lemma}

\begin{proof}
    Prove by induction.
\end{proof}

\begin{lemma}[Transition in Algorithmic Equivalence]
    The algorithmic equivalence relationship is transitive.
    \label{lemma:TransitioninAlgorithmicEquivalence}
    \begin{itemize}
        \item If \( \G \AVS K \E K' @ A \) and \( \G \AVS K' \E K'' @ A \) then \( \G \AVS K \E K'' @ A \).
        \item If \( \G \AVS \tau \E \tau' @ A \) and \( \G \AVS \tau' \E \tau'' @ A \) then \( \G \AVS \tau \E \tau'' @ A \).
        \item If \( \G \AVS M \E M' @ A \) and \( \G \AVS M' \E M'' @ A \) then \( \G \AVS M \E M'' @ A \).
    \end{itemize}
\end{lemma}

\begin{proof}
    Prove by the induction on derivations.  In this lemma, there are two
    hypothesis \( \mathcal{J}_1 \) and \( \mathcal{J}_2 \).  The last rules of
    \( \mathcal{J}_1 \) and \( \mathcal{J}_2 \) are the same.  Because these
    are algorithmic derivation so the last rule of derivation is decided
    uniquely by the shape of judgments.  Then we can prove this theorem easily
    from the induction hypothesis and the last rule.
\end{proof}



\begin{lemma}[Equivalence Preservation of Algorithmic Reduction]
    \label{lemma:EquivalencePreservationOfAlgorithmicReduction}
    This is an algorithmic version of Lemma~\ref{lemma:EquivalencePreservation}.
    \begin{itemize}
        \item If \( \G \V M : \tau @ A \) and \( M \RA M' \) then \( \G \V M \E M' : \tau @ A \).
        \item If \( \G \V \tau :: K @ A \) and \( \tau \RA \tau' \), then \( \G \V \tau \E \tau' :: K \).
    \end{itemize}
\end{lemma}
\begin{proof}
    By induction on the type or kind derivation tree. We show the different case from Lemma~\ref{lemma:EquivalencePreservation}.

    \begin{rneqncase}{\TCsp}{
            \G\V \%_\alpha M:\tau@{A\alpha} \text{ is derived from }
            \G\V M:\tau@A \text{ and }\G\V \tau::*@{A\alpha}
        }
        \begin{itemize}
                \item \textsc{A-\%} \\
                    \( \%_\alpha M \RA M \) when \( \text{FV}(\%_\alpha M) =
                    \emptyset \). In this case \( M \) is a closed term
                    therefore contains no free variables. So, we can derive \(
                    \G \V M : \tau @ A\alpha \) and \( \G \V \%_\alpha M \E M @
                    A\alpha \) from \QPercent.
        \end{itemize}
    \end{rneqncase}
\end{proof}

\begin{lemma}[Type Preservation of Algorithmic Reduction]
    \label{lemma:TypePreservationofAlgorithmicReduction}
    If \( \G \V M:\tau @A \) and \( M \RA M' \) then \( \G \V M' : \tau @ A \).
\end{lemma}

\begin{proof}
    Similar to Theorem \ref{theorem:TypePreservation}, we prove by using
    Lemma~\ref{lemma:AgreementofAlgorithmicTyping} to the second item of
    Lemma~\ref{lemma:EquivalencePreservationOfAlgorithmicReduction}.
\end{proof}



\begin{corollary}[ANF and Term Equivalence]
    \label{corollary:ANFandTermEquivalence}
    If \( \G \V M : \tau @ A \) then \( \G \V M \E \ANF(M) : \tau @ A\).
\end{corollary}
\begin{proof}
    By induction on the length of reduction using
    Lemma~\ref{lemma:EquivalencePreservationOfAlgorithmicReduction}. We must use Lemma~\ref{lemma:SymmetricityinAlgorithmicEquivalence} for the base case.
\end{proof}

\begin{theorem}[Soundness of Algorithmic Typing]
    If a term is typed in algorithmic rules then the term is typed in normal rules, too.
    \begin{itemize}
        \item If \(\G\AVS K \iskind @ A \) then \(\G\V K \iskind @A \).
        \item If \(\G\AVS \tau :: K @ A \) then \(\G\V \tau ::K  @ A \).
        \item If \(\G\AVS M:\tau @ A \) then \(\G\V M:\tau @ A \).
        \item If \(\G\AVS K,K' @ A\) and \(\G\AVS K\E K' @ A \) then \(\G\V K\E K' @ A \).
        \item If \( \G \AVS \tau :: K @ A \), \( \G \AVS \tau' :: K' @ A \), \( \G \AVS K \E K' @ A\) and \(\G\AVS \tau \E \tau' @ A \) then \(\G\V \tau \E \tau' :: K @ A \).
        \item If \(\G \AVS M : \tau @ A\), \( \G \AVS M' : \tau' @ A \), \( \G \AVS \tau \E \tau' @ A\) and \(\G\AVS M\E M' @ A \) then \(\G\V M\E M' :: \tau @ A \).
    \end{itemize}
\end{theorem}

\begin{proof}
    We can prove using induction on the derivation tree.
    \begin{rneqncase}{\KAApp{}}{
            \G\AVS \sigma\ M::K[x\mapsto M]@A \\
            \text{ is derived from } 
            \G\AVS \sigma:: (\Pi x:\tau.K)@A \text{ , }
            \G\AVS M:\tau'@A \text{ and } 
            \G\AVS \tau \E \tau' @ A.
        }
        Considering the derivation of \( \G\AVS \sigma:: (\Pi x:\tau.K)@A \),
        \( \G \AVS \tau::* \). From
        Lemma~\ref{lemma:AgreementofAlgorithmicTyping} on \( \G\AVS M:\tau'@A
        \), \( \G \AVS \tau' :: * @ A \). Using the induction hypothesis on \(
        \G \AVS \tau, \tau'::* @ A \) and \( \G\AVS \tau \E \tau' @ A \), we
        get \( \G\V \tau \E \tau' :: * @ A \). From the induction hypothesis,
        \( \G\V \sigma:: (\Pi x:\tau.K)@A \) and \( \G\V M:\tau'@A \). \TConv
        and \KApp guide to \( \G\V \sigma\ M::K[x\mapsto M]@A \).
    \end{rneqncase}
    \begin{rneqncase}{\TAApp}{
            \G\AVS M\ N : \tau[x\mapsto N]@A \\
            \text{ is derived from } 
            \G\AVS M:(\Pi (x:\sigma).\tau) \text{ , }
            \G\AVS N:\sigma'@A \text{ and }
            % \G\AVS \sigma'::*@A \text{ and }
            \G\AVS \sigma\E\sigma' @A
        }
        Considering the derivation of \( \G\AVS M: (\Pi x:\sigma.\tau)@A \), \(
        \G \AVS \sigma :: * @ A \). Use
        Lemma~\ref{lemma:AgreementofAlgorithmicTyping} on \( \G\AVS N:\sigma'@A
        \) get \( \G \AVS \sigma' :: * @ A \).  Just same as \KApp case, \(
        \G\V \sigma\E\sigma' :: * @A \).  From the induction hypothesis, \(
        \G\V M:(\Pi (x:\sigma).\tau) @ A\) and \( \G\V N:\sigma'@A \). Then, \(
        \G\V M\ N : \tau[x\mapsto N]@A \) from \TConv and \TApp.
    \end{rneqncase}
    \begin{rneqncase}{\QAANF}{
            \G \AVS M \E N@A
            \text{ is derived from } \\
            \ANF(M) \E_\alpha \ANF(N),
            \G \AVS M : \tau @ A \text { and }
            \G \AVS N : \tau' @ A.
        }
        From Lemma~\ref{lemma:AgreementofAlgorithmicTyping}, \( \G \AVS \tau ::
        * @ A \) and \( \G \AVS \tau' :: * @ A \).  Use the induction
        hypothesis on \( \G \AVS M : \tau @ A \) and \( \G \AVS N : \tau' @ A
        \), respectively, get \( \G \V M : \tau @ A \) and \( \G \V N : \tau' @
        A \). Use Corollary~\ref{corollary:ANFandTermEquivalence} on \( \G \V M
        : \tau @ A \)  get \( \G \V M \E \ANF(M) : \tau @ A \).  From
        Lemma~\ref{lemma:Agreement}, \( \G \V M : \tau @ A \) and \( \G \V
        \ANF(M) : \tau @ A \).  Use
        Corollary~\ref{corollary:ANFandTermEquivalence}, \( \G \V N \E \ANF(N)
        : \tau' @ A \). From Lemma~\ref{lemma:Agreement}, \( \G \V N : \tau' @
        A \) and \( \G \V \ANF(N) : \tau' @ A \).  Because \( \ANF(M) \E_\alpha
        \ANF(N) \), \( \G \V \ANF(M) : \tau @ A \) and \( \G \V \ANF(N) : \tau'
        @ A \), \( \G \V \tau \E \tau' @ A \) from
        Lemma~\ref{lemma:TypesOfTheSameTerm}.  Then, using \TConv, we get \( \G \V N
        : \tau @ A \). Finally, we can prove \( \G \V M \E N@A \) from
        Corollary~\ref{corollary:ANFandTermEquivalence}, \QTrans and \QSym.
    \end{rneqncase}
\end{proof}

\begin{lemma}[Free Variable and Stage]
    \label{lemma:FreeVariableandStage}
    If \( \G\V M:\tau@{A\alpha} \) and \( \G\V M:\tau @A \) then \( \FV(M) = \emptyset \).
\end{lemma}

\begin{proof}
    Prove by contradiction. Assume \( \G\V M:\tau@{A\alpha} \) and \( \G\V
    M:\tau @A \) and \( \FV(M) \neq \emptyset \). Then, \( x : \tau @ B, x:
    \tau @ B\alpha \in \G \). However, from well-formedness of \( \G \), there
    is no such \( x \).
\end{proof}

\begin{theorem}[Completeness of Algorithmic Typing]
    \label{theorem:CompletenessofAlgorithmicTyping}
    \begin{itemize}
        \item If \(\G\V K \) then \(\G\AVS K \).
        \item If \(\G\V \tau ::K @ A \) then there is \(K'\) such that \( \G \AVS K' @ A \) and \( \G \AVS K\E K' @ A \) and \( \G\AVS \tau :: K' @ A \).
        \item If \(\G\V M:\tau @ A \) then there is \(\tau'\) such that \( \G \AVS \tau' :: * @ A \) and \( \G \AVS \tau \E \tau' @ A \) and \( \G\AVS M:\tau' @ A \).
        \item If \(\G\V K\E K' @ A \) then \(\G\AVS K\E K' @ A \).
        \item If \(\G\V \tau \E \tau' :: K @ A \) then \(\G\AVS \tau \E \tau' @ A \).
        \item If \(\G\V M\E M' : \tau @ A \) then \(\G \AVS M \E M' @ A \).
    \end{itemize}
\end{theorem}

\begin{proof}
    Prove by induction on the derivation.
    \begin{rneqncase}{\KApp}{
            \G\V \sigma\ M::K[x\mapsto M]@A \\
            \text{ is derived from } 
            \G\V \sigma:: (\Pi x:\tau.K)@A \text{ and } \G\V M:\tau@A.
        }
        Apply the induction hypothesis on \( \G\V \sigma:: (\Pi x:\tau.K)@A \),
        there is a kind \( J \) such that \( \G\AVS J @A \), \( \G\AVS \sigma
        :: J \) and \( \G\AVS J \E (\Pi x:\tau.K)@A \).  The last rule of the
        derivation of \( \G\AVS J \E (\Pi x:\tau.K)@A \) is \QTAAbs, so \( J =
        \Pi x:\tau'.K' \), \( \G \AVS \tau \E \tau' @ A \) and \( \G, x:\tau
        \AVS K \E K' @ A \).  Use Lemma \ref{lemma:Agreement} and the induction
        hypothesis on \( \G \V M : \tau @ A \), \( \G \AVS \tau :: * @ A \).
        Now, we can apply \KAApp to \( \G \AVS \sigma :: \Pi x:\tau'.K' @ A \),
        \( \G \AVS M: \tau @ A \) and \( \G \AVS \tau \E \tau' @ A\) and get \(
        \G \AVS \sigma\ M :: K'[x \mapsto M] @ A \).  From Lemma
        \ref{lemma:TermSubstitutionLemmaOfAlgorithmicJudgement}, \( \G \AVS K[x
        \mapsto M] \E K'[x \mapsto M] @ A \).
    \end{rneqncase}
    \begin{rneqncase}{\KConv}{
            \G\V \tau::J@A \\
            \text{ is derived from }
            \G\V \tau::K@A \text{ and }
            \G\V K\equiv J@A
        }
        Apply the induction hypothesis on \( \G\V \tau::K@A \),
        there is a kind K' such that \( \G \AVS K' @ A \) and 
        \( \G \AVS K\E K' @ A \) and \( \G\AVS \tau :: K' @ A \).
        Apply the induction hypothesis on \( \G\V K\equiv J@A \),
        \( \G\AVS K\equiv J@A \).
        From Lemma \ref{lemma:SymmetricityinAlgorithmicEquivalence} and
        \ref {lemma:TransitioninAlgorithmicEquivalence}, \( \G \AVS J \E K' @ A \).
    \end{rneqncase}
    \begin{rneqncase}{\QTApp}{
            \G\V \tau\ M \E \sigma\ N :: K[x \mapsto M]@A \\
            \text{ is derived from }
            \G\V \tau \E \sigma :: (\Pi x:\rho.K)@A \text{ and }
            \G\V M \E N : \rho @A
        }
        From the induction hypothesis, \( \G\AVS \tau \E \sigma@A \) and \(
        \G\AVS M \E N @A \). Use \TAApp, \( \G\AVS \tau\ M \E \sigma\ N \).
    \end{rneqncase}
    \begin{rneqncase}{\TConv}{
            \G\V M:\tau'@A \\
            \text{ is derived from }
            \G\V M:\tau@A \text{ and }
            \G\V \tau\equiv \tau':: K@A
        }
        Apply the induction hypothesis on \( \G\V \tau\equiv \tau':: K@A \), \(
        \G\AVS \tau\equiv \tau' @A \).  Use Lemma \ref{lemma:Agreement} and the
        induction hypothesis to \( \G\V M : \tau @A \), we get \( \G\AVS \tau
        :: * @ A \). Now we can see \( \tau \) as \( \tau' \) in the second
        item of Theorem \ref{theorem:CompletenessofAlgorithmicTyping}.
    \end{rneqncase}
    \begin{rneqncase}{\QPercent}{
            \G\V \%_\alpha M \E M : \tau@{A\alpha}
            \text{ is derived from }
            \G\V M:\tau@{A\alpha} \text{ and }
            \G\V M:\tau@A
        }
        From Lemma~\ref{lemma:FreeVariableandStage}, \( \FV(M) = \emptyset \).
        Then, \( \%_\alpha M \RA M \) and \( \ANF(\%_\alpha M) = \ANF(M) \).
        Using the induction hypothesis on \( \G\V M:\tau@{A\alpha} \), there is
        a \( \tau' \) such that \( \G \AVS \tau' :: * @ A\alpha \), \( \G \AVS
        \tau \E \tau' @ A\alpha \) and \( \G \AVS M : \tau' @ A\alpha \).
        Using the induction hypothesis on \( \G\V M:\tau@A \), there is a \(
        \tau'' \) such that \( \G \AVS \tau'' :: * @ A \), \( \G \AVS \tau \E
        \tau'' @ A \) and \( \G \AVS M : \tau'' @ A \).  Using \TACsp on \( \G
        \AVS M : \tau'' @ A \), we get \( \G \AVS \%_\alpha M : \tau'' @
        A\alpha \).  Use \QAANF on \( \ANF(\%_\alpha M) = \ANF(M) \), \( \G
        \AVS \%_\alpha M : \tau'' @ A\alpha \) and \( \G \AVS M : \tau'' @ A
        \), we get \( \G \AVS \%_\alpha M \E M @ {A\alpha} \)
    \end{rneqncase}
    \begin{rneqncase}{\QBeta}{
            \G\V(\lambda x:\sigma.M)\ N\E M[x\mapsto N] : \tau[x \mapsto N]@A
            \text{ is derived from }
            \G,x:\sigma@A\V M:\tau@A \text{ and }
            \G\V N:\sigma@A.
        }
        \( \ANF((\lambda x:\sigma.M)\ N) \E_\alpha \ANF(M[x\mapsto N]) \)
        because \( (\lambda x:\sigma.M)\ N \longrightarrow_A M[x\mapsto N] \).
        Using the induction hypothesis on \( \G,x:\sigma@A\V M:\tau@A \), there
        is \( \tau' \) such that \( \G,x:\sigma@A \AVS \tau' :: * @ A \) and \(
        \G,x:\sigma@A \AVS \tau \E \tau' @ A \) and \( \G,x:\sigma@A\AVS
        M:\tau'@A \). Using the induction hypothesis on \( \G\V N:\sigma@A \),
        there is \( \sigma' \) such that \( \G \AVS \sigma' :: * @ A \), \( \G
        \AVS \sigma \E \sigma' @ A \) and \( \G \AVS N : \sigma' \). From
        \TAApp, \( \G \AVS M\ N : \tau'[x \mapsto N]@A \). From
        Lemma~\ref{lemma:TermSubstitutionLemmaOfAlgorithmicJudgement}, \( \G
        \AVS M[x \mapsto N] : \tau'[x \mapsto N] \). Then, we can use \QAANF.
    \end{rneqncase}
    \begin{rneqncase}{\QKSym}{}
        It is obvious from Lemma \ref{lemma:SymmetricityinAlgorithmicEquivalence}.
    \end{rneqncase}
    \begin{rneqncase}{\QKRefl}{}
        It is obvious from Lemma \ref{lemma:ReflexivityinAlgorithmicEquivalence}.
    \end{rneqncase}
    \begin{rneqncase}{\QKTrans}{}
        It is obvious from Lemma \ref{lemma:TransitioninAlgorithmicEquivalence}.
    \end{rneqncase}
\end{proof}
