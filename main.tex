% This is samplepaper.tex, a sample chapter demonstrating the
% LLNCS macro package for Springer Computer Science proceedings;
% Version 2.20 of 2017/10/04
%
\newif\ifnotanonymous \notanonymoustrue
\newif\iffullversion \fullversiontrue
\documentclass{jfp1}
\bibliographystyle{jfp}

\usepackage{algorithm}% http://ctan.org/pkg/algorithms
\usepackage{algpseudocode}% http://ctan.org/pkg/algorithmicx
\usepackage{graphicx}
\usepackage{color}
% "amsmath" and "amsthm" package conflicts with the JFP style file. To use
% amsmath, I comment out equation* environment in JFP style file.
% \usepackage{amsthm}
\usepackage{amsmath}
\usepackage{mathtools}
\usepackage{amssymb}
\usepackage{bcprules, proof}
\usepackage{fancybox}
\usepackage{float}
\usepackage{xparse}
\usepackage{lscape}
\usepackage{xspace}
\usepackage{url}
\usepackage{bcpproof}
\usepackage{multicol}

% Used for displaying a sample figure. If possible, figure files should
% be included in EPS format.
%
% If you use the hyperref package, please uncomment the following line
% to display URLs in blue roman font according to Springer's eBook style:
% This package conflicts with JFP style file.
% \usepackage{cite}
% \usepackage[dvipdfmx]{hyperref}
% \renewcommand\UrlFont{\color{blue}\rmfamily}

\newcommand{\red}[1]{\textcolor{red}{#1 }}
\newcommand{\blue}[1]{\textcolor{blue}{#1 }}

\newcommand{\LTP}{$\lambda^{\triangleright\%}$\xspace}
\newcommand{\LMD}{$\lambda^{\textrm{MD}}$\xspace}
\newcommand{\LLF}{$\lambda\textrm{LF}$\xspace}

\newcommand{\G}{\Gamma}
\newcommand{\D}{\Delta}
\newcommand{\V}{\vdash_\Sigma}
\newcommand{\VT}{\vdash\hspace{-.50em}\raisebox{0.28em}{\tiny{$\TB$}}}
\newcommand{\iskind}{\operatorname{\ kind}}
\newcommand{\TW}{{\mathop{\triangleright}}}
\newcommand{\TWL}{{\mathop{\triangleleft}}}
\newcommand{\F}{\forall}
\newcommand{\TB}{{\mathop{\blacktriangleright}}}
\newcommand{\TBL}{{\mathop{\blacktriangleleft}}}
\newcommand{\E}{\equiv}
\newcommand{\FV}{\text{FV}}
\newcommand{\FTV}{\text{FSV}}

\newcommand{\WStar}{\textsc{W-Star}\xspace}
\newcommand{\WAbs}{\textsc{W-Pi}\xspace}
\newcommand{\WPi}{\textsc{W-Pi}\xspace}
\newcommand{\WCsp}{\textsc{W-Csp}\xspace}
\newcommand{\WApp}{\textsc{W-App}\xspace}
\newcommand{\WTW}{\textsc{W-$\TW$}\xspace}

\newcommand{\WAStar}{\textsc{WA-Star}\xspace}
\newcommand{\WAAbs}{\textsc{WA-Abs}\xspace}
\newcommand{\WACsp}{\textsc{WA-Csp}\xspace}
\newcommand{\WAApp}{\textsc{WA-App}\xspace}
\newcommand{\WATW}{\textsc{WA-$\TW$}\xspace}

\newcommand{\KVar}{\textsc{K-Var}\xspace}
\newcommand{\KTConst}{\textsc{K-TConst}\xspace}
\newcommand{\KPi}{\textsc{K-Pi}\xspace}
\newcommand{\KApp}{\textsc{K-App}\xspace}
\newcommand{\KConv}{\textsc{K-Conv}\xspace}
\newcommand{\KTW}{\textsc{K-$\TW$}\xspace}
\newcommand{\KTWL}{\textsc{K-$\TWL$}\xspace}
\newcommand{\KGen}{\textsc{K-Gen}\xspace}
\newcommand{\KCsp}{\textsc{K-Csp}\xspace}

\newcommand{\KAVar}{\textsc{KA-Var}\xspace}
\newcommand{\KATConst}{\textsc{KA-TConst}\xspace}
\newcommand{\KAPi}{\textsc{KA-Pi}\xspace}
\newcommand{\KAApp}{\textsc{KA-App}\xspace}
\newcommand{\KAConv}{\textsc{KA-Conv}\xspace}
\newcommand{\KATW}{\textsc{KA-$\TW$}\xspace}
\newcommand{\KATWL}{\textsc{KA-$\TWL$}\xspace}
\newcommand{\KAGen}{\textsc{KA-Gen}\xspace}
\newcommand{\KACsp}{\textsc{KA-Csp}\xspace}

\newcommand{\TConst}{\textsc{T-Const}\xspace}
\newcommand{\TVar}{\textsc{T-Var}\xspace}
\newcommand{\TAbs}{\textsc{T-Abs}\xspace}
\newcommand{\TApp}{\textsc{T-App}\xspace}
\newcommand{\TConv}{\textsc{T-Conv}\xspace}
\newcommand{\TTB}{\textsc{T-$\TB$}\xspace}
\newcommand{\TTBL}{\textsc{T-$\TBL$}\xspace}
\newcommand{\TGen}{\textsc{T-Gen}\xspace}
\newcommand{\TIns}{\textsc{T-Ins}\xspace}
\newcommand{\TCsp}{\textsc{T-Csp}\xspace}

\newcommand{\TAConst}{\textsc{TA-Const}\xspace}
\newcommand{\TAVar}{\textsc{TA-Var}\xspace}
\newcommand{\TAAbs}{\textsc{TA-Abs}\xspace}
\newcommand{\TAApp}{\textsc{TA-App}\xspace}
\newcommand{\TAConv}{\textsc{TA-Conv}\xspace}
\newcommand{\TATB}{\textsc{TA-$\TB$}\xspace}
\newcommand{\TATBL}{\textsc{TA-$\TBL$}\xspace}
\newcommand{\TAGen}{\textsc{TA-Gen}\xspace}
\newcommand{\TAIns}{\textsc{TA-Ins}\xspace}
\newcommand{\TACsp}{\textsc{TA-Csp}\xspace}

\newcommand{\QKPi}{\textsc{QK-Pi}\xspace}
\newcommand{\QKCsp}{\textsc{QK-Csp}\xspace}
\newcommand{\QKRefl}{\textsc{QK-Refl}\xspace}
\newcommand{\QKSym}{\textsc{QK-Sym}\xspace}
\newcommand{\QKTrans}{\textsc{QK-Trans}\xspace}

\newcommand{\QKAPi}{\textsc{QKA-Pi}\xspace}
\newcommand{\QKACsp}{\textsc{QKA-Csp}\xspace}
\newcommand{\QKARefl}{\textsc{QKA-Refl}\xspace}
\newcommand{\QKASym}{\textsc{QKA-Sym}\xspace}
\newcommand{\QKATrans}{\textsc{QKA-Trans}\xspace}

\newcommand{\QTAbs}{\textsc{QT-Abs}\xspace}
\newcommand{\QTApp}{\textsc{QT-App}\xspace}
\newcommand{\QTTW}{\textsc{QT-$\TW$}\xspace}
\newcommand{\QTGen}{\textsc{QT-Gen}\xspace}
\newcommand{\QTCsp}{\textsc{QT-Csp}\xspace}
\newcommand{\QTRefl}{\textsc{QT-Refl}\xspace}
\newcommand{\QTSym}{\textsc{QT-Sym}\xspace}
\newcommand{\QTTrans}{\textsc{QT-Trans}\xspace}

\newcommand{\QTAAbs}{\textsc{QTA-Abs}\xspace}
\newcommand{\QTAApp}{\textsc{QTA-App}\xspace}
\newcommand{\QTATW}{\textsc{QTA-$\TW$}\xspace}
\newcommand{\QTAGen}{\textsc{QTA-Gen}\xspace}
\newcommand{\QTACsp}{\textsc{QTA-Csp}\xspace}
\newcommand{\QTAConst}{\textsc{QTA-Const}\xspace}
\newcommand{\QTASym}{\textsc{QTA-Sym}\xspace}
\newcommand{\QTATrans}{\textsc{QTA-Trans}\xspace}

\newcommand{\QAbs}{\textsc{Q-Abs}\xspace}
\newcommand{\QApp}{\textsc{Q-App}\xspace}
\newcommand{\QTB}{\textsc{Q-$\TB$}\xspace}
\newcommand{\QTBL}{\textsc{Q-$\TBL$}\xspace}
\newcommand{\QGen}{\textsc{Q-Gen}\xspace}
\newcommand{\QIns}{\textsc{Q-Ins}\xspace}
\newcommand{\QCsp}{\textsc{Q-Csp}\xspace}
\newcommand{\QRefl}{\textsc{Q-Refl}\xspace}
\newcommand{\QSym}{\textsc{Q-Sym}\xspace}
\newcommand{\QTrans}{\textsc{Q-Trans}\xspace}
\newcommand{\QBeta}{\textsc{Q-$\beta$}\xspace}
\newcommand{\QEta}{\textsc{Q-$\eta$}\xspace}
\newcommand{\QTBLTB}{\textsc{Q-$\TBL\TB$}\xspace}
\newcommand{\QLambda}{\textsc{Q-$\Lambda$}\xspace}
\newcommand{\QPercent}{\textsc{Q-\%}\xspace}

\newcommand{\QAAbs}{\textsc{QA-Abs}\xspace}
\newcommand{\QAApp}{\textsc{QA-App}\xspace}
\newcommand{\QAVar}{\textsc{QA-Var}\xspace}
\newcommand{\QAConst}{\textsc{QA-Const}\xspace}
\newcommand{\QATB}{\textsc{QA-$\TB$}\xspace}
\newcommand{\QATBL}{\textsc{QA-$\TBL$}\xspace}
\newcommand{\QAGen}{\textsc{QA-Gen}\xspace}
\newcommand{\QAIns}{\textsc{QA-Ins}\xspace}
\newcommand{\QACsp}{\textsc{QA-Csp}\xspace}
\newcommand{\QARefl}{\textsc{QA-Refl}\xspace}
\newcommand{\QASym}{\textsc{QA-Sym}\xspace}
\newcommand{\QATrans}{\textsc{QA-Trans}\xspace}
\newcommand{\QABeta}{\textsc{QA-$\beta$}\xspace}
\newcommand{\QAEta}{\textsc{QA-$\eta$}\xspace}
\newcommand{\QATBLTB}{\textsc{QA-$\TBL\TB$}\xspace}
\newcommand{\QALambda}{\textsc{QA-$\Lambda$}\xspace}
\newcommand{\QAPercent}{\textsc{QA-\%}\xspace}
\newcommand{\QAANF}{\textsc{QA-ANF}\xspace}

\newcommand{\ID}[1]{\infer[]{#1}{\vdots}}
\newcommand{\MD}[1]{\mathcal{D}_#1}

\newcommand{\I}{\textrm{Int}}
\newcommand{\B}{\textrm{Bool}}
\newcommand{\M}{\textrm{Mat}}

\newcommand{\RWH}{\longrightarrow_{\text{wh}}}
\newcommand{\AER}{\longrightarrow_{\text{AE}}}
\newcommand{\RA}{\longrightarrow_{\text{A}}}
\newcommand{\AV}{\vdash\!\!\!\raisebox{0.4ex}{\scalebox{0.7}{$\TB$}}}
\newcommand{\AVS}{\AV_\Sigma\xspace}
% \newcommand{\EWH}{\E_{\text{wh}}}
% \newcommand{\EA}{\E_{\text{A}}}
\newcommand{\ANF}{\text{ANF}\xspace}



\newcommand{\rulefbox}[1]{\fbox{\ensuremath{#1}} \hspace{1mm}}

\newcommand{\AI}[1]{\textcolor{red}{[#1 -- AI]}}
\newcommand{\AK}[1]{\textcolor{blue}{[#1 -- AK]}}
\newcommand{\RESUME}{\AK{========== RESUME HERE ==========}}

\newtheorem{definition}{Definition}
\newtheorem{lemma}{Lemma}
\newtheorem{theorem}{Theorem}
\newtheorem{corollary}{Corollary}

\begin{document}

\title{A Dependently Typed Multi-Stage Calculus}
\author[Akira Kawata, Igarashi Atsushi]{AKIRA KAWATA, ATSUSHI IGARASHI\\
\textit{Kyoto University, Japan}\\
(\textit{email}:\texttt{akira@fos.kuis.kyoto-u.ac.jp}, \texttt{igarashi@kuis.kyoto-u.ac.jp})
}			% Author name in English

\maketitle

\begin{abstract}

% Merits of multi-stage programming

Multi-stage programming is a programming paradigm that enables generation and
evaluation of code at run-time. For instance, MetaOCaml is a well-known
multi-stage programming language supporting quasi-quotation and cross-stage
persistence for the manipulation of code fragments as first-class values
and an evaluation construct for the execution of programs dynamically
generated by code manipulation. A significant merit of multi-stage
programming is the utilization of run-time information during code
generation. For example, we can generate an efficient vector sum function
for arbitrary fixed-length vectors by unrolling loops.

% Problems of multi-stage programming

However, such optimization may cause severe problems because specialized
functions only can be used with restricted arguments. In the case of a
vector addition function, the generated function is specialized for the
given length and must be used with vectors of the correct length.
Otherwise, the function can access outside of vectors because the
loop-unrolled code never checks the boundaries of its input.

% Our solution for the problem

To resolve the above problem, we study a dependently typed extension of a
multi-stage programming language \`a la MetaOCaml. Dependent types bring to
multi-stage programming the enforcement of strong invariants---beyond
simple type safety---on the behavior of dynamically generated code. In the
case of the vector sum function, dependent types restrict the input
arguments to vectors of the correct length. However, extending multi-stage
programming with dependent types is non-trivial because such a type system
must integrate stages of types, which is the number of surrounding
quotations.

% Our contibution

To rigorously study properties of such an extension, we develop \LMD, which is
an extension of typed calculus \LTP proposed by Hanada and Igarashi with
dependent types, and prove its properties, including preservation,
confluence, strong normalization for full reduction, and progress for
staged reduction.  Furthermore, we design algorithmic typing to implement a
type-inference program and prove its equivalence to the original typing. 

\end{abstract}

\tableofcontents

% !TEX root = ../main.tex

\section{Introduction}
\label{sec:intro}

\subsection{Multi-stage Programming and MetaOCaml}

% What is multi-stage programming?

Multi-stage programming makes it easier for programmers to implement generation
and execution of code at run time by providing language constructs for
composing and running pieces of code as first-class values. A promising
application of multi-stage programming is (run-time) code specialization, which
generates program code specialized to partial inputs to the program and such
applications are studied in the
literature~\cite{Kiselyov2018Reconcilong,Mainland2012MetaHaskell,Taha2007Gentle}.

% MetaOCaml

MetaOCaml~\cite{CalcagnoTahaHuangLeroy2003Implementing,Kiselyov2014BERMetaOCamlBER} is an extension of
OCaml\footnote{\url{http://ocaml.org}} with special constructs for multi-stage
programming, including brackets and escape, which are (hygienic)
quasi-quotation, and \texttt{run}, which is similar to \texttt{eval} in Lisp,
and cross-stage persistence (CSP)~\cite{MetaML}.  Programmers can easily write
code generators by using these features.  Moreover, MetaOCaml is equipped with
a powerful type system for safe code generation and execution.  The notion of
code types is introduced to prevent code values that represent ill-typed
expressions from being generated.  For example, a quotation of expression
\texttt{1 + 1} is given type \texttt{int code} and a code-generating function,
which takes a code value \(c\) as an argument and returns \(c \texttt{ + } c\),
is given type \texttt{int code -> int code} so that it cannot be applied to,
say, a quotation of \texttt{"Hello"}, which is given type \texttt{string code}.
Ensuring safety for \verb|run| is more challenging because code types by
themselves do not guarantee that the execution of code values never results in
unbound variable errors.  Taha and Nielsen~\cite{TahaNielsen2003Environment}
introduced the notion of environment classifiers to address the problem,
developed a type system to ensure not only type-safe composition but also
type-safe execution of code values, and proved a type soundness theorem (for a
formal calculus \(\lambda^\alpha\) modeling a pure subset of MetaOCaml).

% The problem of existing multi-stage programming type system

However, the type system, which is based on the Hindley--Milner
polymorphism~\cite{Milner78JCSS}, is not strong enough to guarantee invariant
beyond simple types.  For example, Kiselyov~\cite{Kiselyov2018Reconcilong} demonstrates
specialization of vector/matrix computation with respect to the sizes of
vectors and matrices in MetaOCaml but the type system of MetaOCaml cannot
prevent such specialized functions from being applied to vectors and matrices
of different sizes.

\subsection{Multi-stage Programming with Dependent Types}

% 依存型とは何か?

One natural idea to address this problem is the introduction of dependent types
to express the size of data structures in static types~\cite{XiPfenning1998Eliminating}.  For
example, we could declare vector types indexed by the size of vectors as
follows.
\begin{verbatim}
    Vector :: Int -> *
\end{verbatim}
\verb|Vector| is a type constructor that takes an integer (which represents the
length of vectors): for example, \verb|Vector 3| is the type for vectors whose
lengths are 3.  Then, we hope to specialize vector/matrix functions for their
size and get a piece of function code, whose type respects the given size,
\emph{provided at specialization time}.  For example, we would like to
specialize a function to add two vectors for the size of vectors, that is, to
implement a code generator that takes a (nonnegative) integer $n$ as an input
and generates a piece of function code of type 
\verb|(Vector |$n$\verb| -> Vector |$n$\verb| -> Vector |$n$\verb|) code|.

\subsection{Our Work}
In this paper, we develop a new multi-stage calculus \LMD by extending the
existing multi-stage calculus \LTP\cite{HanadaIgarashi2014CSP} with dependent types and
study its properties.  We base our work on \LTP, in which the four multi-stage
constructs are handled slightly differently from MetaOCaml, because its type
system and semantics are arguably simpler than
\(\lambda^\alpha\)~\cite{TahaNielsen2003Environment}, which formalizes the design of
MetaOCaml more faithfully.  Dependent types are based on \LLF~\cite{benjamin2005attapldependent},
which has one of the simplest forms of dependent types.  Our technical
contributions are summarized as follows:
\begin{itemize}
    \item We give a formal definition of \LMD with its syntax, type system and
        two kinds of reduction: full reduction, allowing reduction of any redex,
        including one under $\lambda$-abstraction and quotation, and staged reduction, a
        small-step call-by-value operational semantics that is closer to the intended
        multi-stage implementation.
    \item We show preservation, strong normalization, and confluence for full
        reduction; and show unique decomposition (and progress as its
        corollary) for staged reduction.
    \item We give algorithmic typing for \LMD, which is essential to implement
        a type checker and show the equivalence between the algorithmic typing
        and the original typing. Especially, we proved that two equalities are
        equivalence. One is defined with a combination of equivalence rules,
        and another is defined by comparing the normal forms of two terms.
\end{itemize}
The combination of multi-stage programming and dependent types has been
discussed by Pasalic, Taha, and Sheard~\cite{PasalicTahaSheard2002Tagless} and Brady and
Hammond~\cite{BradyHammond2006Dependently} but, to our knowledge, our work is a first
formal calculus of full-spectrum dependently typed multi-stage programming with
all the key constructs mentioned above.

This article extends the previous work of the
authors~\cite{kawata2019dependently} with the algorithmic typing in
Section~\ref{sec:algorithmic-typing-and-equality} and its equivalence to the
original typing rules.

\subsubsection{Organization of the Paper}

The organization of this paper is as follows.
Section~\ref{sec:informal-overview} gives an informal overview of
\LMD. Section~\ref{sec:formal-definition} defines \LMD and
Section~\ref{sec:properties} shows properties of \LMD.
Section~\ref{sec:related-work} discusses related work and Section
\ref{sec:conclusion} concludes the paper with discussion of future
work.


% !TEX root = ../main.tex

\section{Informal Overview of \LMD \label{sec:informal-overview}}

We describe our calculus \LMD informally.  \LMD is based on
\LTP~\cite{HanadaIgarashi2014CSP} by Hanada and Igarashi and so we start with a review of
\LTP.

\subsection{\LTP}

% quote and unquote

In \LTP, brackets (quasi-quotation) and escape (unquote) are written
$\TB_\alpha M$ and $\TBL_\alpha M$, respectively.  For example,
$\TB_\alpha (1 + 1)$ represents code of expression $1 + 1$ and thus
evaluates to itself.  Escape $\TBL_\alpha M$ may appear under
$\TB_\alpha$; it evaluates $M$ to a code value and splices it into the
surrounding code fragment.  Such splicing is expressed by the
following reduction rule:
\begin{align*}
	\TBL_\alpha (\TB_\alpha M) \longrightarrow M .
\end{align*}

% Stage and run

The subscript $\alpha$ in $\TB_\alpha$ and $\TBL_\alpha$ is a \textit{stage
variable}\footnote{ In Hanada and Igarashi~\cite{HanadaIgarashi2014CSP}, it was called a
\textit{transition variable}, which is derived from correspondence to modal
logic, studied by Tsukada and Igarashi~\cite{Tsukada}.} and a sequence of stage
variables is called a \textit{stage}.  Intuitively, a stage represents the
depth of nested brackets.  Stage variables can be abstracted by
$\Lambda\alpha.M$ and instantiated by an application $M\ A$ to stages.  For
example, $\Lambda\alpha.\TB_\alpha ((\lambda x:\I.x+10)\ 5)$ is a code value,
where \(\alpha\) is abstracted.  If it is applied to \(A = \alpha_1 \cdots
\alpha_n\), \(\TB_\alpha\) becomes \(\TB_{\alpha_1} \cdots \TB_{\alpha_n}\); in
particular, if \(n = 0\), \(\TB_\alpha\) disappears.  So, an application of
$\Lambda\alpha.\TB_\alpha ((\lambda x:\I.x+10)\ 5)$ to the empty sequence
\(\varepsilon\) reduces to (unquoted) \((\lambda x:\I.x+10)\ 5\) and to 15.  In
other words, application of a \(\Lambda\)-abstraction to $\varepsilon$
corresponds to \texttt{run}.  This is expressed by the following reduction
rule:
\begin{align*}
	(\Lambda\alpha.M)\ A \longrightarrow M[\alpha\mapsto A]
\end{align*}
where stage substitution \([\alpha \mapsto A]\) manipulates the nesting of
\(\TB_\alpha\) and \(\TBL_\alpha\) (and also \(\%_\alpha\) as we see later).

% CSP

Cross-stage persistence (CSP), which is an essential feature of \LTP, is a
primitive operator to embed values (not necessarily code values) into a code
value. For example, a \LTP-term
\[
  M_1 = \lambda x:\I.\Lambda\alpha.(\TB_\alpha ((\%_\alpha x) * 2))
\]
takes an integer \(x\) as an input and returns a code value, into
which \(x\) is embedded.  If $M_1$ is applied to $38 + 4$ as in
\[
  M_2 = (\lambda x:\I.\Lambda\alpha.(\TB_\alpha ((\%_\alpha x) * 2)))\ (38 + 4),
\]
then it evaluates to \(M_3 = \Lambda\alpha.(\TB_\alpha ((\%_\alpha 42) * 2))\).
According to the semantics of \LTP, the subterm $\%_\alpha 42$ means that it
waits for the surrounding code to be run (by an application to $\varepsilon$)
and so it does not reduce further.  If \(M_3\) is run by application to
\(\varepsilon\), substitution of \(\varepsilon\) for \(\alpha\) eliminates
\(\TB_\alpha\) and \(\%_\alpha\) and so \(42 * 2\), which reduces to 84, is
obtained.  CSP is practically important because one can call library functions
from inside quotations.

% Type system of \LTP

The type system of \LTP uses code types---the type of code of type
\(\tau\) is written \(\TW_\alpha \tau\)---for typing \(\TB_\alpha\),
\(\TBL_\alpha\) and \(\%_\alpha\).  It takes stages into account: a
variable declaration (written $x:\tau@A$) in a type environment is associated with its
declared stage $A$ as well as its type $\tau$ and the type judgement of \LTP is of
the form $\G \vdash M : \tau@A$, in which $A$ stands for the stage
of term $M$.\footnote{%
  In Hanada and Igarashi~\cite{HanadaIgarashi2014CSP}, it is written
  $\G \vdash^A M : \tau$.
  }
For example,
$y:\I@\alpha \vdash (\lambda x:\I.y) : \I \to \I @ \alpha$ holds, but
$y:\I@\alpha \vdash (\lambda x:\I.y) : \I \to \I @ \varepsilon$ does
not because the latter uses $y$ at stage \(\varepsilon\) but $y$ is
declared at $\alpha$.  Quotation \(\TB_\alpha M\) is given type
\(\TW_\alpha \tau\) at stage \(A\) if \(M\) is given type \(\tau\) at
stage \(A\alpha\); unquote \(\TBL_\alpha M\) is given type \(\tau\)
at stage \(A\alpha\) if \(M\) is given type \(\TW_\alpha \tau\) at
stage \(A\alpha\); and CSP \(\%_\alpha M\) is give type \(\tau\)
at stage \(A\alpha\) if \(M\) is given type \(\tau\) at \(A\).
These are expressed by the following typing rules.
\begin{center}
	\infrule{\G\vdash M:\tau @{A\alpha}}{\G\vdash \TB_{\alpha}M:\TW_{\alpha}\tau @A} \hfil
	\infrule{\G\vdash M:\TW_{\alpha}\tau @A}{\G\vdash \TBL_{\alpha}M:\tau @{A\alpha}} \hfil
	\infrule{\G\vdash M: \tau @A}{\G\vdash \%_{\alpha}M:\tau @{A\alpha}}
\end{center}

\subsection{Extending \LTP with Dependent Types}

% General explanation of extension

In this paper, we add a simple form of dependent types---{\`a} la Edinburgh
LF~\cite{harper1993framework} and \LLF~\cite{attapl}---to \LTP.  Types can be
indexed by terms as in \texttt{Vector} in Section~\ref{sec:intro} and
\(\lambda\)-abstractions can be given dependent function types of the form
\(\Pi x:\tau. \sigma\) but we do not consider type operators (such as
$\texttt{list } \tau$) or abstraction over type variables.  We introduce kinds
to classify well-formed types and equivalences for kinds, types, and terms---as
in other dependent type systems---but we have to address a question of how the
notion of stages (should) interact with kinds and types.

% Issues around type and stage

On the one hand, base types such as \(\I\) should be able to be used at every
stage as in \LTP so that \(\lambda x:\I.\Lambda \alpha. \TB_\alpha \lambda
y:\I.M\) is a valid term (here, \(\I\) is used at \(\varepsilon\) and
\(\alpha\)). Similarly for indexed types such as Vector 4. On the other hand,
it is not immediately clear how a type indexed by a variable, which can be used
only at a declared stage, can be used. For example, consider
\[\TB_\alpha (\lambda x:\I. (\TBL_\alpha (\lambda y:\text{Vector
  }x.M)N) )
  \text{ and }
  \lambda x:\I. \TB_\alpha (\lambda y:\text{Vector }x.M).
\]
Is Vector\ \(x\) a legitimate type at \(\varepsilon\) (and \(\alpha\), resp.)
even if \(x:\I\) is declared at stage \(\alpha\) (and \(\varepsilon\), resp.)?
We will give our answer to this question in two steps.

% Type-level constants and stages

First, type-level constants such as \(\I\) and Vector can be used at every
stage in \LMD. Technically, we introduce a signature that declares kinds of
type-level constants and types of constants.  For example, a signature for the
Boolean type and constants is given as follows $\B::*, \text{true}:\B,
\text{false}:\B$ (where $*$ is the kind of proper types). Declarations in a
signature are not associated to particular stages; so they can be used at every
stage.

% Indexed type and stage

Second, an indexed type such as Vector\ 3 or Vector\ $x$ is well formed only at
the stage(s) where the index term is well-typed.  Since constant \(3\) is
well-typed at every stage (if it is declared in the signature), Vector\ 3 is a
well-formed type at every stage, too.  However, Vector\ $M$ is well-formed only
at the stage where index term $M$ is typed.  Thus, the kinding judgment of \LMD
takes the form \(\G\V \tau :: K @ A\), where stage $A$ stands for where
\(\tau\) is well-formed.  For example, given \(\text{Vector}:: \I \rightarrow
*\) in the signature \(\Sigma\), \(x:\I@\varepsilon \V \text{Vector }x :: *
@\varepsilon\) can be derived but neither \(x:\I@\alpha \V \text{Vector }x :: *
@\varepsilon\) nor \(x:\I@\varepsilon \V \text{Vector }x :: * @\alpha\) can be.

% Using CSP in indexed type

Apparently, the restriction above sounds too severe, because a term like
\(\lambda x:\I. \TB_\alpha (\lambda y:\text{Vector }x.M) \), which models a
typical code generator which takes the size $x$ and returns code for vector
manipulation specialized to the given size, will be rejected. It seems crucial
for \(y\) to be given a type indexed by $x$. We can address this problem by
CSP---In fact, $\text{Vector }x$ is not well formed at $\alpha$ under
$x:\I@\varepsilon$ but $\text{Vector }(\%_\alpha x)$ is!  Thus, we can still
write \(\lambda x:\I. \TB_\alpha (\lambda y:\text{Vector }(\%_\alpha x).M) \)
for the typical sort of code generators.

% Foreshadow for \QPercent

Our decision that well-formedness of types takes stages of index terms into
account leads to the introduction of CSP at the type level and special
equivalence rules, as we see later.


% !TEX root = ../main.tex

\section{Formal Definition of \LMD \label{sec:formal-definition}}
\label{sec:formal}

In this section, we give a formal definition of \LMD, including its syntax,
full reduction, and type system.  In addition to the full reduction, in which
any redex at any stage can be reduced, we also give staged reduction, which
models program execution (at \(\varepsilon\)-stage).

\subsection{Syntax}

We assume the denumerable set of \emph{type-level constants}, ranged over by
metavariables \(X, Y, Z\), the denumerable set of \emph{variables}, ranged
over by \(x,y,z\), the denumerable set of \emph{constants}, ranged over by
\(c\), and the denumerable set of \emph{stage variables}, ranged over by
\(\alpha, \beta, \gamma\).  The metavariables \(A, B, C\) range over
sequences of stage variables; we write \(\varepsilon\) for the empty
sequence. \LMD is defined by the following grammar:

{%\small
\begin{align*}
   \textrm{kinds}             &  & K,J,I,H,G                & ::= * \mid \Pi x:\tau.K                                                           \\
    \textrm{types}             &  & \tau,\sigma,\rho,\pi,\xi & ::= X \mid \Pi x:\tau.\sigma \mid \tau\ M \mid \TW_{\alpha} \tau \mid \F\alpha.\tau \\
    \textrm{terms}             &  & M,N,L,O,P                & ::= c \mid x \mid \lambda x:\tau.M\ \mid M\ N \mid \TB_\alpha M                   \\
                               &  &                          & \ \ \ \ \mid \TBL_\alpha M \mid \Lambda\alpha.M \mid M\ A \mid \%_\alpha M        \\
    \textrm{signatures}         &  & \Sigma                   & ::= \emptyset \mid \Sigma, X::K \mid \Sigma, c:\tau                               \\
    \textrm{type env.} &  & \Gamma                   & ::= \emptyset \mid  \Gamma,x:\tau @A                                              \\
\end{align*}
}

% Kinds

A kind, which is used to classify types, is either $*$, the kind of
proper types (types that terms inhabit), or $\Pi x\colon\tau.K$, the kind
of type operators that takes $x$ as an argument of type $\tau$ and returns a type
of kind $K$.

% Types

A type is a type-level constant $X$, which is declared in the signature with
its kind, a dependent function type $\Pi x:\tau.\sigma$, an application $\tau\
M$ of a type (operator of $\Pi$-kind) to a term, a code type $\TW_\alpha \tau$,
or an $\alpha$-closed type $\F\alpha.\tau$.  An example of an application of a
type (operator) of $\Pi$-kind to a term is $\text{Vector}\ 10$; it is well
kinded if, say, the type-level constant $\text{Vector}$ has kind $\Pi x:\I.*$.
A code type $\TW_\alpha \tau$ is for a code fragment of a term of type $\tau$.
An $\alpha$-closed type, when used with $\TW_\alpha$, represents runnable code.

% Terms

Terms include ordinary (explicitly typed) \(\lambda\)-terms, constants, whose
types are declared in signature $\Sigma$, and the following five forms related
to multi-stage programming: $\TB_\alpha M$ represents a code fragment;
$\TBL_\alpha M$ represents escape; $\Lambda\alpha.M$ is a stage variable
abstraction; $M\ A$ is an application of a stage abstraction $M$ to stage $A$;
and $\%_\alpha M$ is an operator for cross-stage persistence.

% Signature

We adopt the tradition of \LLF-like systems, where types of constants and
kinds of type-level constants are globally declared in a signature $\Sigma$,
which is a sequence of declarations of the form $c:\tau$ and $X::K$. For
example, when we use Boolean in \LMD, $\Sigma$ includes $\B :: *,
\textrm{true}:\B, \textrm{false}:\B$. Type environments are sequences of
triples of a variable, its type, and its stage. We write
\(\textit{dom}(\Sigma)\) and \(\textit{dom}(\Gamma)\) for the set of
(type-level) constants and variables declared in \(\Sigma\) and \(\Gamma\),
respectively. As in other multi-stage
calculi~\cite{TahaNielsen2003Environment,TsukadaIgarashi2010Logical,HanadaIgarashi2014CSP}, a variable declaration
is associated with a stage so that a variable can be referenced only at the
declared stage. On the contrary, constants and type-level constants are
\emph{not} associated with stages; so, they can appear at any stage. We
define well-formed signatures and well-formed type environments later.

% Free and bound variables

The variable $x$ is bound in $M$ by $\lambda x:\tau.M$ and in $\sigma$ by $\Pi
x:\tau.\sigma$, as usual; the stage variable $\alpha$ is bound in $M$ by
$\Lambda \alpha.M$ and $\tau$ by $\F\alpha.\tau$.  The notion of free variables
is defined in a standard manner.  We write $\FV(M)$ and $\FTV(M)$ for the set
of free variables and the set of free stage variables in $M$, respectively.
Similarly, $\FV(\tau)$, $\FTV(\tau)$, $\FV(K)$, and $\FTV(K)$ are defined.  We
sometimes abbreviate $\Pi x:\tau_1.\tau_2$ to $\tau_1 \rightarrow \tau_2$ if
$x$ is not a free variable of $\tau_2$.  We identify $\alpha$-convertible terms
and assume the names of bound variables are pairwise distinct.

% Primitives of multi-stage programming

The prefix operators $\TW_\alpha, \TB_\alpha, \TBL_\alpha$, and $\%_\alpha$ are
given higher precedence over the three forms $\tau\ M$, $M\ N$, $M\ A$ of
applications, which are left-associative. The binders $\Pi$, $\forall$, and
$\lambda$ extend as far to the right as possible.  Thus, $\F\alpha.\TW_{\alpha}
(\Pi x:\I.\text{Vector}\ 5)$ is interpreted as $\F\alpha.(\TW_{\alpha} (\Pi
x:\I.(\text{Vector}\ 5)))$; and $\Lambda\alpha.\lambda x:\I.\TB_\alpha x\ y$
means $\Lambda\alpha.(\lambda x:\I.(\TB_\alpha x)\ y)$.

% Remark on restriction of \LMD

\paragraph{Remark:} Basically, we define \LMD to be an extension of \LTP with
dependent types. One notable difference is that \LMD has only one kind of
\(\alpha\)-closed types, whereas \LTP has two kinds of \(\alpha\)-closed types
\(\forall\alpha.\tau\) and \(\forall^\varepsilon\alpha.\tau\).  We have omitted
the first kind, for simplicity, and dropped the superscript $\varepsilon$ from
the second. It would not be difficult to recover the distinction to show
properties related to program residualization~\cite{HanadaIgarashi2014CSP}, although they
are left as conjectures.

\subsection{Reduction}

% Substitutions

Next, we define full reduction for \LMD.  Before giving the definition of
reduction, we define two kinds of substitutions. Substitution $M[x\mapsto N],
\tau[x \mapsto N]$ and $K[x \mapsto N]$ are ordinary capture-avoiding
substitution of term $N$ for $x$ in term $M$, type $\tau$, and kind $K$,
respectively, and we omit their definitions here.  Substitution $M[\alpha
\mapsto A], \tau [\alpha \mapsto A], K[\alpha \mapsto A]$ and $B[\alpha\mapsto
A]$ are substitutions of stage $A$ for stage variable $\alpha$ in term $M$,
type $\tau$, kind $K$, and stage $B$, respectively.  We show representative
cases below.

\begin{align*}
    (\lambda x:\tau.M)[\alpha \mapsto A] & = \lambda x:(\tau[\alpha \mapsto A]).(M[\alpha \mapsto A])                                  \\
    (M\ B)[\alpha \mapsto A]             & = (M[\alpha \mapsto A])\ B[\alpha\mapsto A]                                                 \\
    (\TB_\beta M)[\alpha \mapsto A]      & = \TB_{\beta[\alpha \mapsto A]}M[\alpha \mapsto A]                                          \\
    (\TBL_\beta M)[\alpha \mapsto A]     & = \TBL_{\beta[\alpha \mapsto A]}M[\alpha \mapsto A]                                         \\
    (\%_\beta M)[\alpha \mapsto A]       & = \%_{\beta[\alpha \mapsto A]}M[\alpha \mapsto A]                                           \\
    (\beta B)[\alpha \mapsto A]          & = \beta (B[\alpha\mapsto A])                               & (\text{if } \alpha \neq \beta) \\
    (\beta B)[\alpha \mapsto A]          & = A (B[\alpha\mapsto A])                                   & (\text{if } \alpha = \beta)
\end{align*}

% Series of stage variables

Here, $\TB_{\alpha_1\cdots\alpha_n} M$, $\TBL_{\alpha_1\cdots\alpha_n} M$, and
$\%_{\alpha_1\cdots\alpha_n} M$ $(n \geq 0)$ stand for $\TB_{\alpha_1} \cdots
\TB_{\alpha_n} M$, $\TBL_{\alpha_n}\cdots \TBL_{\alpha_1} M$, and
$\%_{\alpha_n}\cdots \%_{\alpha_1} M$, respectively.  In particular,
$\TB_{\varepsilon} M = \TBL_{\varepsilon} M = \%_{\varepsilon} M = M$.  Also,
it is important that the order of stage variables is reversed for $\TBL$ and
$\%$.  We also define substitutions of a stage or a term for variables in type
environment $\G$.

% Reduction

\begin{definition}[Reduction]
    The relations $M \longrightarrow_\beta N$, $M \longrightarrow_\blacklozenge N$, and $M \longrightarrow_\Lambda N$ are the least compatible relations closed under the rules below.
\begin{center}
  \begin{align*}
     & (\lambda x:\tau.M) N \longrightarrow_\beta M[x \mapsto N]         \\
     & \TBL_\alpha (\TB_\alpha M)\longrightarrow_\blacklozenge M         \\
     & (\Lambda \alpha.M)\ A \longrightarrow_\Lambda M[\alpha \mapsto A]
  \end{align*}
\end{center}
\end{definition}

We write \( M \longrightarrow M' \) iff \( M \longrightarrow_{\beta} M' \), \(
M \longrightarrow_\blacklozenge \) or \( M \longrightarrow_\Lambda \).

The relation $\longrightarrow_\beta$ represents ordinary $\beta$-reduction in
the \(\lambda\)-calculus; the relation $\longrightarrow_\blacklozenge$
represents that quotation $\TB_\alpha M$ is canceled by escape and $M$ is
spliced into the code fragment surrounding the escape; the relation
$\longrightarrow_\Lambda$ means that a stage abstraction applied to stage $A$
reduces to the body of the abstraction where $A$ is substituted for the stage
variable.  There is no reduction rule for CSP as with Hanada and Igarashi
\cite{HanadaIgarashi2014CSP}.  The CSP operator $\%_\alpha$ disappears when $\varepsilon$
is substituted for $\alpha$.

We show an example of a reduction sequence in
Figure~\ref{fig:example-of-reduction}. Underlines show the redexes.  Please
note that terms in type annotations are also reduced like
Figure~\ref{fig:reduction-in-type-annotations}.

\begin{figure}[tbp]
\begin{align*}
    & \hspace{10mm} \underline{(\lambda f:\I\to\I.(\Lambda\alpha.\TB_\alpha (\%_\alpha f\ 1 + (\TBL_\alpha \TB_\alpha 3))\ \varepsilon))\ (\lambda x:\I.x)} \\
    & \longrightarrow_\beta (\Lambda\alpha.\TB_\alpha (\%_\alpha (\lambda x:\I.x)\ 1 + (\underline{\TBL_\alpha \TB_\alpha 3})))\ \varepsilon        \\
    & \longrightarrow_\blacklozenge \underline{(\Lambda\alpha.\TB_\alpha (\%_\alpha (\lambda x:\I.x)\ 1 + 3))\ \varepsilon}                                         \\
    & \longrightarrow_\Lambda \underline{(\lambda x:\I.x)\ 1} + 3                                                                                           \\
    & \longrightarrow_\beta 1 + 3                                                                                                                           \\
    & \longrightarrow^* 4
\end{align*}
    \caption{Example of reduction.}
    \label{fig:example-of-reduction}
\end{figure}

\begin{figure}[tbp]
    \begin{align*}
        & \hspace{10mm} \lambda v:(\text{Vec }(\underline{1+3})).v \\
        & \longrightarrow_\beta \lambda v:(\text{Vec }4).v
    \end{align*}
    \caption{Example of reduction in a type annotation.}
    \label{fig:reduction-in-type-annotations}
\end{figure}

\subsection{Type System}

In this section, we define the type system of \LMD.  It consists of eight
judgment forms for signature well-formedness, type environment well-formedness,
kind well-formedness, kinding, typing, kind equivalence, type equivalence, and
term equivalence.  We list the judgment forms in
Figure~\ref{fig:LMD-six-judgments}. They are all defined in a mutually
recursive manner. We will discuss each judgment below.

\begin{figure}[tbp]
    \begin{center}
    \begin{align*}
      \vdash & \Sigma                     & \text{signature well-formedness}        \\
      \V     & \G                         & \text{type environment well-formedness} \\
      \G     & \V K \iskind @ A           & \text{kind well-formedness}             \\
      \G     & \V \tau :: K @ A           & \text{kinding}                          \\
      \G     & \V M : \tau @ A            & \text{typing}                           \\
      \G     & \V K \E J @ A              & \text{kind equivalence}                 \\
      \G     & \V \tau \E \sigma :: K @ A & \text{type equivalence}                 \\
      \G     & \V M \E N : \tau @ A       & \text{term equivalence}
    \end{align*}
    \caption{Eight judgment forms of the type system of \LMD.}
    \label{fig:LMD-six-judgments}
  \end{center}
\end{figure}

\subsubsection{Signature and Type Environment Well-formedness}
The rules for Well-formed signatures and type environments are
shown below:
%
{\small
\begin{center}
  \infrule{
  }{
    \vdash \emptyset
  }
  \hfil
  \infrule{
    \vdash \Sigma \andalso
    \V K \iskind @ \varepsilon \\
    X\notin\textit{dom}(\Sigma)
  }{
    \vdash \Sigma, X::K
  }
  \hfil
  \infrule{
    \vdash \Sigma \andalso
    \V \tau :: * @ \varepsilon \\
    c\notin\textit{dom}(\Sigma)
  }{
    \vdash \Sigma, c:\tau
  }
  \\[2mm]
  \infrule{
  }{
    \V \emptyset
  }
  \hfil
  \infrule{
    \V \Gamma \andalso
    \Gamma \V \tau :: * @ A \andalso
    x\notin\textit{dom}(\Sigma)
  }{
    \V \Gamma, x:\tau@A
  }
\end{center}
}

To add declarations to a signature, the kind/type of a (type-level) constant
has to be well-formed at stage \(\varepsilon\) so that it is used at any stage.
In what follows, well-formedness is not explicitly mentioned, but we assume
that all signatures and type environments are well-formed.

\subsubsection{Kind Well-formedness and Kinding}

The rules for kind well-formedness and kinding are a straightforward adaptation
from \LLF and \LTP. So, we omit the definitions.

\subsubsection{Typing}
\label{sec:typing}

\begin{figure}[tbp]
  \begin{center}
    \infrule[\TConst]{c:\tau \in \Sigma}{\G \V c:\tau @A} \hfil
    \infrule[\TVar]{x:\tau @A \in \G}{\G \V x:\tau @A} \\[2mm]
    \infrule[\TAbs]{\G\V \sigma::*@A\andalso\G,x:\sigma@A\V M:\tau @A}{\G\V(\lambda (x:\sigma).M):(\Pi (x:\sigma).\tau)@A} \\[2mm]
    \infrule[\TApp]{\G\V M:(\Pi (x:\sigma).\tau)@A \andalso \G\V N:\sigma@A}{\G\V M\ N : \tau[x\mapsto N]@A} \\[2mm]
    \infrule[\TConv]{\G\V M:\tau @A \andalso \G\V \tau\equiv \sigma :: K@A}{\G\V M:\sigma@A} \\[2mm]
    \infrule[\TTB]{\G\V M:\tau @{A\alpha}}{\G\V\TB_{\alpha}M:\TW_{\alpha}\tau @A} \andalso
    \infrule[\TTBL]{\G\V M:\TW_{\alpha}\tau @A}{\G\V\TBL_{\alpha}M:\tau @{A\alpha}} \\[2mm]
    \infrule[\TGen]{\G\V M:\tau @A \andalso \alpha\notin\rm{FTV}(\G)\cup\rm{FTV}(A)}{\G\V\Lambda\alpha.M:\forall\alpha.\tau @A} \\[2mm]
    \infrule[\TIns]{\G\V M:\forall\alpha.\tau @A}{\G\V M\ B:\tau[\alpha \mapsto B]@A} \andalso
    \infrule[\TCsp]{
        \G\V M:\tau@A \andalso \G\V \tau::*@{A\alpha}
    }{
      \G\V \%_\alpha M:\tau@{A\alpha}
    }
    \caption{Typing Rules.}
    \label{fig:typing-rules}
  \end{center}
\end{figure}

The typing rules of \LMD are shown in Figure~\ref{fig:typing-rules}.
The rule \TConst{} means that a constant can appear at any stage.
The rules \TVar, \TAbs, and \TApp{} are almost the same as those in the simply typed
lambda calculus or \LLF.  Additional conditions are that subterms must be
typed at the same stage (\TAbs{} and \TApp); the type
annotation/declaration on a variable has to be a proper type of kind
$*$ (\TAbs) at the stage where it is declared (\TVar{} and \TAbs).

% \TConv

As in standard dependent type systems, \TConv{} allows us to replace the type
of a term with an equivalent one. For example, assuming integers and
arithmetic, a value of type $\textrm{Vector}\ (4+1)$ can also have type
$\textrm{Vector}\ 5$ because of \TConv{}.

% Typing rules for a multi-stage calculus

The rules \TTB, \TTBL, \TGen, \TIns, and \TCsp{} are constructs for multi-stage
programming. \TTB{} and \TTBL{} are the same as in \LTP, as we explained in
Section \ref{sec:informal-overview}. The rule \TGen{} for stage abstraction is
straightforward. The condition $\alpha\notin\rm{FTV}(\G)\cup\rm{FTV}(A)$
ensures that the scope of $\alpha$ is in $M$, and avoids capturing variables
elsewhere. The rule \TIns{} is for applications of stages to stage
abstractions. The rule \TCsp{} is for CSP, which means that, if term $M$ is of
type $\tau$ at stage $A$ and type \( \tau \) is also legitimate at stage \(
A\alpha \), then $\%_\alpha M$ is of type $\tau$ at stage $A\alpha$. Note that
CSP is also applied to the type \(\tau\) (although it is implicit) in the
conclusion. Thanks to implicit CSP, the typing rule is the same as in \LTP.

\subsubsection{Kind, Type and Term Equivalence}

Since the syntax of kinds, types, and terms is mutually recursive, the
corresponding notions of equivalence are also mutually recursive.  They are
congruences closed under a few axioms for term equivalence.  Thus, the rules
for kind and type equivalences are not very interesting, except that implicit
CSP is allowed.  We show a few representative rules below.

\begin{center}
    \infrule[\textsc{QK-Csp}]{%
        \G\V K \E J @ A
        }{
            \G\V K \E J @ A\alpha
        }
        \hfil
        \infrule[\QTCsp]{
            \G\V \tau \E \sigma :: *@A
        }{
            \G\V \tau \E \sigma :: *@{A\alpha}
        }
        \\[2mm]
        \infrule[\QTApp]{%
            \G\V \tau \E \sigma :: (\Pi x:\rho.K)@A \andalso
            \G\V M \E N : \rho @A
            }{
                \G\V \tau\ M \E \sigma\ N :: K[x \mapsto M]@A
            }
\end{center}

We show the rules for term equivalence in
Figure~\ref{fig:term-equivalence-rules}, omitting straightforward rules for
reflexivity, symmetry, transitivity, and compatibility.  The rules \QBeta,
\QTBLTB, and \QLambda{} correspond to $\beta$-reduction,
$\blacklozenge$-reduction, and $\Lambda$-reduction, respectively.

\begin{figure}[tbp]
    \begin{center}
   \infrule[\QBeta]{\G,x:\sigma@A\V M:\tau @A \andalso \G\V N:\sigma@A}{\G\V(\lambda x:\sigma.M)\ N\E M[x\mapsto N] : \tau[x \mapsto N]@A} \\[2mm]
    \infrule[\QLambda]{\G\V (\Lambda\alpha.M) : \forall\alpha.\tau @A}{\G\V (\Lambda\alpha.M)\ \varepsilon \E M[\alpha \mapsto \varepsilon] : \tau[\alpha \mapsto \varepsilon]@A} \\[2mm]
    \infrule[\QTBLTB]{\G\V M \E N : \tau @A}{\G\V \TBL_\alpha(\TB_\alpha M) \E N : \tau @A} \hfil
    \infrule[\QPercent]{\G\V M:\tau @{A\alpha} \andalso \G\V M:\tau @A}{\G\V\%_\alpha M \E M : \tau @{A\alpha}}
    \caption{Term Equivalence Rules.}
    \label{fig:term-equivalence-rules}
  \end{center}
\end{figure}

% \QPercentの説明

The only rule that deserves elaboration is the last rule \QPercent.
Intuitively, it means that the CSP operator applied to term $M$ can be
removed if $M$ is also well-typed at the next stage \(A\alpha\).
For example, constants do not depend on the stage (see \TConst) and
so \(\G\V \%_\alpha c \E c : \tau @ A\alpha\) holds but variables
do depend on stages and so this rule does not apply.

\subsubsection{Example}

We show an example of a dependently typed code generator in a
hypothetical language based on \LTP.  
This language provides definitions by \textbf{let},
recursive functions (represented by \textbf{fix}), \textbf{if}-expressions,
and primitives cons, head, and tail to manipulate vectors. We assume that
$\text{cons}$ is of type $\Pi n:\I.\I \to \text{Vector}\ n \to \text{Vector}\ (n+1)$, 
$\text{head}$ is of type $\Pi n:\I.\text{Vector}\ (n+1) \to \I$, and
$\text{tail}$ is of type $\Pi n:\I.\text{Vector}\ (n+1) \to (\text{Vector}\ n)$.

Let's consider an application, for example, in computer graphics, in which we
have potentially many pairs of vectors of the fixed (but statically unknown)
length and a function---such as vector addition---to be applied to
them. This function should be fast because it is applied many times and be
safe because just one runtime error may ruin the whole long-running calculation.

\newcommand{\Vpn}{\text{Vector}\ (\%_\alpha n)}

Our goal is to define the function vadd of type
\[
  \Pi n:\I.\F\beta.\TW_\beta(\Vpn\to\Vpn\to\Vpn).
\]
\renewcommand{\Vpn}{\text{Vector}\ n}
It takes the length $n$ and returns ($\beta$-closed) code of a
function to add two vectors of length $n$.  The generated
code is run by applying it to \(\varepsilon\) to obtain
a function of type \(\Vpn\to\Vpn\to\Vpn\) as expected.

We start with the helper function vadd$_1$, which takes a stage, the length $n$
of vectors, and two quoted vectors as arguments and returns code that computes
the addition of the given two vectors:
\begin{tabbing}
	  $\textbf{let}\ \text{vadd}_1 : \F\alpha.\Pi n:\I.\TW_\alpha\Vpn\to\TW_\alpha\Vpn\to\TW_\alpha\Vpn$                                \\
	  \hspace{6mm} \= $= \textbf{fix}\ f.\Lambda\alpha.\lambda n:\I.\ \lambda v_1:\TW_\alpha\Vpn.\ \lambda v_2:\TW_\alpha\Vpn.$            \\
	  \> \hspace{6mm} \= $\textbf{if}\ n = 0 \ \textbf{then}\ \TB_\alpha \text{nil}$ \\
	  \>\> $\textbf{else}\ \TB_\alpha ($ \= $\textbf{let}\ t_1 = \text{tail}\ (\TBL_\alpha v_1)\ \textbf{in}$ \\
	  \>\>\> $\textbf{let}\ t_2 = \text{tail}\ (\TBL_\alpha v_2)\ \textbf{in}$ \\
          \>\>\> $\text{cons}\ $\= $(\text{head}\ (\TBL_\alpha v_1) + \text{head} \ (\TBL_\alpha v_2))$ \\
          \>\>\>\> $\TBL_\alpha (f\ (n-1)\ (\TB_\alpha t_1)\ (\TB_\alpha t_2)))$
\end{tabbing}
Note that the generated code does not contain branching on $n$ or recursion.
(Here, we assume that the type system can determine whether $n=0$ when
\textbf{then}- and \textbf{else}-branches are type-checked so that both
branches can be given type \(\TW_\alpha \text{Vector }n\).)

Using vadd$_1$, the main function vadd can be defined as follows:
\renewcommand{\Vpn}{\text{Vector}\ (\%_\beta n)}
\begin{align*}
	  & \textbf{let}\ \text{vadd}: \Pi n:\I.\F\beta.\TW_\beta(\Vpn\to\Vpn\to\Vpn)                \\ 
	  & \hspace{6mm} = \lambda n:\I.\Lambda\beta.\TB_\beta (\lambda v_1:\Vpn.\ \lambda v_2:\Vpn. \\
	  & \hspace{63mm} \TBL_\beta (\text{vadd}_1\ \beta\ n\ (\TB_\beta\ v_1)\ (\TB_\beta\ v_2))) 
\end{align*}
\renewcommand{\Vpn}{\text{Vector}\ (\%_\beta 5)}%
The auxiliary function vadd$_1$ generates code to compute addition of
the formal arguments $v_1$ and $v_2$ without branching on $n$ or recursion.
As we mentioned already, if this function is applied to a
(nonnegative) integer constant, say 5, it returns function code for adding
two vectors of size 5.  The type of vadd\ 5, obtained by
substituting 5 for $n$, is
\( \F\beta.\TW_\beta(\Vpn\to\Vpn\to\Vpn) \).
\renewcommand{\Vpn}{\text{Vector}\ 5}
If the obtained code is run by applying to \(\varepsilon\),
the type of vadd\ 5\ \(\varepsilon\) is
\(\Vpn\to\Vpn\to\Vpn\) as expected.

There are other ways to implement the vector addition function: by using tuples
instead of lists if the length for all the vectors is statically known or by
checking the lengths of lists for every pair dynamically.  However, our method
is better than these alternatives in two points.  First, our function,
$\text{vadd}_1$ can generate functions for vectors of arbitrary length unlike
the one using tuples.  Second, $\text{vadd}_1$ has an advantage in speed over
the one using dynamic checking because it can generate an optimized function
for a given length.

\paragraph{Remark:}
If the generated function code is composed with another piece of code of type, say,
\(\TW_\gamma \text{Vector }5\), \QPercent{} plays an essential role; that is,
\(\text{Vector }5\) and \(\text{Vector }(\%_\gamma 5)\), which would occur
by applying the generated code to \(\gamma\) (instead of \(\varepsilon\)), are syntactically
different types but \QPercent{} enables to equate them.
Interestingly, Hanada and Igarashi~\cite{HanadaIgarashi2014CSP} rejected the idea of
reduction that removes $\%_\alpha$ when they developed \LTP{}, as such
reduction does not match the operational behavior of the CSP operator in
implementations. However, as an equational system for multi-stage programs,
the rule \QPercent{} makes sense.

\subsection{Algorithmic Typing}

% What is algorithmic typing?

Algorithmic typing rules are rules for the implementation of type inference
program which takes an environment \( \Gamma \), a signature \( \Sigma \), a
term \( M \), and a stage \( A \) and returns a type \( \tau \) which satisfies
\( \G \V M : \tau @ A \). If there is no such \( \tau \), the algorithm tells
failure.  We already defined typing rules for \LMD in Section \ref{sec:typing},
but they include some non-syntax directed rules. Non-syntax directed rules mean
their premise cannot be decided from the goal. When we implement the type
inference program of \LMD, it is necessary to make up type derivation trees
from bottom to up.  So, we need to make all the typing rules syntax-directed so
that algorithmic typing is equivalent to the original typing. Using this
algorithmic typing rules, we can make a type inference algorithm easily just by
applying the unique appropriate rule one by one. When there is no such rule,
the algorithm returns failure. We show the equivalence ot the original typing
rules and the algorithmic typing rules in Section \ref{sec:properties}.

\begin{algorithm}
\caption{Type inference algorithm}\label{euclid}
\begin{algorithmic}[1]
    \Procedure{type-inference}{$\G, \Sigma, M, A$}
    \If{$M=c$}
        \If{there is $\tau$ such that $c:\tau$ is in $\Sigma$}
        \State \textbf{return} $\tau$
        \Else
        \State \textbf{return} \textbf{Fail}
        \EndIf
    \EndIf
\EndProcedure
\end{algorithmic}
\end{algorithm}


% Judgements

In algorithmic typing, there are 8 judgments as with original typing in Figure
\ref{fig:LMD-six-judgments}, i.e., well-formed signature, well-formed
environment, well-formed kinding, kinding, typing, kind equivalence, type
equivalence, and term equivalence. We replace \( \V \) with \( \AV \) in
judgments to distinguish them from judgments of the original typing and omit
kinds and types from typing equivalence and term equivalence, respectively. The
definitions of well-formed signature, well-formed environment, and well-formed
kinding are the same with original typing, so we don't explain them here.

% Kinding rules and typing rules

Kinding and typing rules are almost the same as the original typing, but there
are no conversion rules, such as \TConv and \KConv. Conversion rules are not
syntax-directed because we can apply them at any typing or kinding judgments.
Thus, we must remove them from algorithmic typing.  As you can see in
\cite{benjamin2005attapldependent}, all uses of conversion rules in a dependent type system can be
put into application rules. The simplification is possible because only
application rules check the equivalence between the actual argument and formal
argument, but all other rules don't check the equality of types. Therefore, the
equivalence of types becomes a problem only in application rules. Then, if we
replace type equality checks in application rules with a conversion check, we
can eliminate conversion rules from algorithmic typing. The two application
rules are as follows.

\begin{center}
    {\small
    \infrule[{\KAApp}]{
        \G\AVS \sigma:: (\Pi x:\tau.K)@A \andalso 
        \G\AVS M:\tau'@A \\
        \G\AVS \tau' :: * @ A \andalso 
        \G\AVS \tau \E \tau' @ A
    }{
        \G\AVS \sigma\ M::K[x\mapsto M]@A
    } \hfil
    \infrule[\TAApp]{
        \G\AVS M:(\Pi (x:\sigma).\tau) \andalso
        \G\AVS N:\sigma'@A \\
        \G\AVS \sigma' :: * @ A \andalso
        \G\AVS \sigma\E\sigma' @A
    }{
        \G\AVS M\ N : \tau[x\mapsto N]@A
    } 
    }
\end{center}

% Kind and type equivalence rules

Unlike the original rules, the kind and type equivalence rules of the
algorithmic typing have no rules for transitivity, symmetry, and reflexivity.
We omitted these rules because they are non-syntax directed. Instead of
including them in the rules, we prove transitivity, symmetry, and reflexivity
as meta theorems.

% Term equivalence rules -- especially on ANF

Dissimilar to all other rules of the algorithmic typing, the term equivalence
rules of algorithmic typing are completely different from the original ones. We
replaced all rules of term equivalence with \QAANF. Although the original
equivalence rules check the equivalence of terms by constructing a derivation
tree, the algorithmic equivalence rules calculate the normal form of two terms
and compare them with just \( \alpha \)-equivalence. \( \ANF(M) \) is the
normal form of \( M \) under algorithmic reduction as defined in Figure
\ref{fig:algorithmic-reduction}. Algorithmic reduction is composed of normal
reductions of \LMD, which are \textsc{(A-$\beta$)}, \textsc{(A-$\TBL\TB$)}, and
\textsc{(A-$\Lambda$)}, and one special reduction \textsc{(A-$\%$)}, which
corresponds to \QPercent in the normal term equivalence rules. Just the same as
the normal reduction \( M \longrightarrow M' \), we define algorithmic
reduction \( M \longrightarrow_{\text{A}} \).  

\begin{center}
    \infrule[\QAANF]{ \ANF(M) \E_\alpha \ANF(N) }{\G \AV M \E N@A}
\end{center}

\begin{figure}[tbp]
    \begin{center}
        \begin{align*}
            & (\lambda x:\tau.M) N \RA M[x \mapsto N]       &  & \textsc{(A-$\beta$)}   \\
            & \TBL_\alpha (\TB_\alpha M) \RA M              &  & \textsc{(A-$\TBL\TB$)} \\
            & (\Lambda \alpha.M)\ A \RA M[\alpha \mapsto A] &  & \textsc{(A-$\Lambda$)} \\
            & \%_\alpha M \RA M \qquad (\emptyset = \text{FV}(\%_\alpha M)) &  & \textsc{(A-$\%$)}
        \end{align*}
    \end{center}
    \caption{Algorithmic Basic Reduction}
    \label{fig:algorithmic-reduction}
\end{figure}

% % Type inference algorithm
%
% Now, we can construct an algorithm for type inference using this algorithmic
% typing. In a conventional pseudocode notation, the algorithm looks like this:
%
% \begin{tabbing}
%     gettype(\( \Gamma, \Sigma, M, A \)) = if there is a algorithmic 
% \end{tabbing}

\subsection{Staged Semantics}

The reduction given above is full reduction and any redexes---even under
$\TB_\alpha$---can be reduced in an arbitrary order.  Following previous
work~\cite{HanadaIgarashi2014CSP}, we introduce (small-step, call-by-value) staged
semantics, where only $\beta$-reduction or $\Lambda$-reduction at stage
$\varepsilon$ or the outer-most $\blacklozenge$-reduction are allowed, modeling
an implementation.

We start with the definition of values. Since terms under quotations are
not executed, the grammar is indexed by stages.

\begin{definition}[Values]
  The family $V^A$ of sets of values, ranged over by $v^A$,
  is defined by the following grammar.  In the grammar, $A' \neq \varepsilon$ is assumed.
  \begin{align*}
    v^\varepsilon \in V^\varepsilon & ::= \lambda x:\tau.M \mid\ \TB_\alpha v^\alpha \mid \Lambda\alpha.v^\varepsilon                                             & \\
    v^{A'} \in V^{A'}               & ::= x \mid \lambda x:\tau.v^{A'} \mid v^{A'}\ v^{A'} \mid\ \TB_\alpha v^{A'\alpha} \mid \Lambda\alpha.v^{A'} \mid v^{A'}\ B & \\
                                    & \quad\   \mid\ \TBL_\alpha v^{A''} (\text{if } A' = A''\alpha \text{ for some } \alpha, A'' \neq \varepsilon)               & \\
                                    & \quad\   \mid\ \%_\alpha v^{A''} (\text{if } A' = A''\alpha)
  \end{align*}
\end{definition}

Values at stage $\varepsilon$ are $\lambda$-abstractions, quoted pieces of
code, or $\Lambda$-abstractions. The body of a $\lambda$-abstraction can be any
term, but the body of $\Lambda$-abstraction has to be a value.  It means that
the body of $\Lambda$-abstraction must be evaluated.  The side condition for
$\TBL_\alpha v^{A'}$ means that escapes in a value can appear only under nested
quotations because an escape under a single quotation splices the code value
into the surrounding code. See Hanada and Igarashi~\cite{HanadaIgarashi2014CSP} for
details.

In order to define staged reduction, we define redex and evaluation contexts.

\begin{definition}[Redex]
  The sets of $\varepsilon$-redexes (ranged over by $R^\varepsilon$) and $\alpha$-redexes (ranged over by $R^\alpha$) are defined by the following grammar.
  \begin{align*}
     & R^\varepsilon ::= (\lambda x:\tau.M)\ v^\varepsilon \mid (\Lambda\alpha.v^\varepsilon)\ \varepsilon \\
     & R^\alpha      ::=\ \TBL_\alpha \TB_\alpha v^\alpha                                                         \\
  \end{align*}
\end{definition}

\begin{definition}[Evaluation Context]
  Let $B$ be either \(\varepsilon\) or a stage variable \(\beta\).
  The family of sets $ECtx^A_B$ of evaluation contexts, ranged over by $E^A_B$, is defined by the following grammar (in which $A'$ stands for a non-empty stage).
  %  $A$ is assumed to be nonempty (but $B,A'$ can be empty).
  \begin{align*}
    E^\varepsilon_B \in ECtx^\varepsilon_B & ::= \square\ (\text{if\ } B = \varepsilon)
    \mid E^\varepsilon_B\ M \mid v^\varepsilon\ E^\varepsilon_B \mid \TB_\alpha E^\alpha_B
    \mid \Lambda\alpha.E^\varepsilon_B \mid E^\varepsilon_B\ A                                                                                    \\
    E^{A'}_B \in ECtx^{A'}_B               & ::= \square\ (\text{if } A' = B) \mid \lambda x:\tau.E^{A'}_B \mid E^{A'}_B\ M \mid v^{A'}\ E^{A'}_B \\
                                           & \mid \TB_\alpha E^{A'\alpha}_B \mid \TBL_\alpha E^{A}_B \ (\text{where } A\alpha = A')               \\
                                           & \mid \Lambda\alpha.E^{A'}_B \mid E^{A'}_B\ A \mid \%_\alpha\ E^{A}_B \ (\text{where } A\alpha = A')
  \end{align*}
\end{definition}

The subscripts $A$ and $B$ in $E^A_B$ stand for the stage of the evaluation
context and of the hole, respectively. The grammar represents that staged
reduction is left-to-right and call-by-value, and terms under \(\Lambda\) are
reduced. Terms at non-$\varepsilon$ stages are not reduced, except redexes of
the form \(\TBL_\alpha \TB_\alpha v^\alpha\) at stage \(\alpha\).  A few
examples of evaluation contexts are shown below:
\begin{align*}
  \square\ (\lambda x:\I.x)                  & \in  ECtx^\varepsilon_\varepsilon \\
  \Lambda\alpha.\square\ \varepsilon            & \in ECtx^\varepsilon_\varepsilon  \\
  \TBL_\alpha \TB_\alpha \TBL_\alpha \square & \in ECtx^\alpha_\varepsilon
\end{align*}
We write $E^A_B[M]$ for the term obtained by filling the hole $\square$ in $E^A_B$ by $M$.

Now we define staged reduction using the redex and evaluation contexts.

\begin{definition}[Staged Reduction]\sloppy
  The staged reduction relation, written $M \longrightarrow_s N$, is defined by
  the least relation closed under the rules below.
  \begin{align*}
    E^A_\varepsilon [(\lambda x:\tau.M)\ v^\varepsilon] & \longrightarrow_s E^A_\varepsilon[M[x\mapsto v^\varepsilon]]      \\
    E^A_\varepsilon [(\Lambda\alpha.v^\varepsilon)\ A]  & \longrightarrow_s E^A_\varepsilon[v^\varepsilon[\alpha\mapsto A]] \\
    E^A_\alpha [\TBL_\alpha \TB_\alpha v^\alpha]        & \longrightarrow_s E^A_\alpha[v^\alpha]                            \\
  \end{align*}
\end{definition}

This reduction relation reduces a term in a deterministic, left-to-right,
call-by-value manner. An application of an abstraction is executed only at
stage \(\varepsilon\) and only a quotation at stage \(\varepsilon\) is spliced
into the surrounding code---notice that, if \(\TB_\alpha v^\alpha\) is at stage
\(\varepsilon\), then the redex \(\TBL_\alpha \TB_\alpha v^\alpha\) is at stage
\(\alpha\). In other words, terms in brackets are not evaluated until the terms
are run and arguments of a function are evaluated before the application.  We
show an example of staged reductions. Underlines show the redexes.
\begin{align*}
                    & (\Lambda\alpha.(\TB_\alpha \underline{\TBL_\alpha \TB_\alpha ((\lambda x:\I.x)\ 10))})\ \varepsilon \\
  \longrightarrow_s & \underline{(\Lambda\alpha.(\TB_\alpha ((\lambda x:\I.x)\ 10)))\ \varepsilon}                        \\
  \longrightarrow_s & \underline{(\lambda x:\I.x)\ 10}                                                                    \\
  \longrightarrow_s & 10
\end{align*}


% !TEX root = ../main.tex

\section{Properties of \LMD \label{sec:properties}}

\subsection{Properties of Typing}

In this subsection, we show the basic properties of \LMD: preservation, strong
normalization, confluence for full reduction, and progress for staged
reduction.

First, we start from the Weakening Lemma which we need in the proof of the Substitution Lemma.
We let $\mathcal{J}$ stand for the judgments \({K \iskind @A}\),
\({\tau::K@A}\), \({M:\tau@A}\), \({K \E K' @ A}\), \({\tau \E \tau' @ A}\),
and \({M \E M' : \tau @ A}\).  Substitutions \({\mathcal{J}[z \mapsto M]}\) and
\({\mathcal{J}[\alpha \mapsto A]}\) are defined in a straightforward manner.
We use these notations throughout this paper.

\begin{lemma}[Weakening]
    \label{lemma:Weakening}
    If \(\G \V \mathcal{J}@A\) and \(\G\) is a sub-sequence of \(\D\), then \(\D \V \mathcal{J}@A\).
\end{lemma}

% Substitution Lemma

The Substitution Lemma in \LMD{} is a little more complicated than usual
because there are eight judgment forms and two kinds of substitution.  The Term
Substitution Lemma states that term substitution $[z \mapsto M]$ preserves
derivability of judgments. The Stage Substitution Lemma states a similar
property for stage substitution $[\alpha\mapsto A]$.  

\begin{lemma}[Term Substitution]
    \label{lemma:TermSubstitution}
    If \({\G, z:\xi@B, \D \V \mathcal{J}}\) and \({\G\V N:\xi @B}\), then \({\G, (\D[z \mapsto N]) \V \mathcal{J}[z \mapsto N]}\).  Similarly, if \({\V \G, z:\xi@B, \D}\) and
    \( {\G\V N:\xi @B} \), then ${ \V \G, (\D[z \mapsto N]) }$.
\end{lemma}

\begin{lemma}[Stage Substitution]\
    \label{lemma:StageSubstitution}
    If $\G \V \mathcal{J}$, then $\G[\beta\mapsto B] \V \mathcal{J}[\beta\mapsto B]$.  Similarly, if $\V \G$, then $\V \G[\beta\mapsto B]$.
\end{lemma}

\begin{proof}
    By simultaneous induction on derivations.
\end{proof}

% Agreement

We prove agreement to use later in many proofs.

\begin{lemma}[Agreement]\
    \label{lemma:Agreement}
    \begin{itemize}
        \item If \(\G\V \tau::K@A\), then \(\G\V K \iskind@A \).
        \item If \(\G\V M:\tau@A\), then \(\G\V \tau::*@A\).
        \item If \(\G\V K\E K'@A\), then \(\G\V K \iskind @A\) and \(\G\V K' \iskind @A\).
        \item If \(\G\V \tau\E \tau':: K@A\), then \(\G\V \tau::K@A\) and \(\G\V \tau'::K@A\).
        \item If \(\G\V M\E M' : \tau@A\), then \(\G\V M:\tau@A\) and \(\G\V M':\tau@A\).
    \end{itemize}
\end{lemma}
\begin{proof}
    By induction on the derivation tree.
\end{proof}


% Inversion Lemma

The following Inversion Lemma is needed to prove the main theorems.  As
usual~\cite{TAPL}, the Inversion Lemma enables us to infer the types of
subterms of a term from the shape of the term.

\begin{lemma}[Inversion]\ 
    \label{lemma:Inversion}
	\begin{itemize}
		\item If $\G \V (\lambda x:\sigma.M) : \rho$ then there are $\sigma'$ and $\tau'$ such that
		      $\rho = \Pi x:\sigma'.\tau'$, $\G \V \sigma \E \sigma'@A$ and $\G ,x:\sigma'@A\V M:\tau'@A$.
		\item If $\G \V \TB_\alpha M : \tau@A$ then 
		      there is $\sigma$ such that $\tau = \TW_\alpha \sigma$ and $\G \V M : \sigma@A$.
		  \item If $\G \V \Lambda\alpha.M : \tau$ then 
		  there is $\sigma$ such that $\sigma = \forall\alpha.\sigma$ and $\G \V M : \sigma@A$.% and $\alpha \notin \FTV(\G) \cup \FV(A)$.
	\end{itemize}
\end{lemma}

\begin{proof}
  Each item is strengthened by statements about type equivalence.
    For example, the first statement is augmented by
      If ${\G \V \rho \E (\Pi x:\sigma.\tau) : K @A}$, then there exist
      $\sigma'$ and $\tau'$ such that ${\rho = \Pi x:\sigma'.\tau'}$ and
      ${\G \V \sigma \E \sigma' : K @A}$ and
      ${\G, x:\sigma@A\V \tau \E \tau' : J @A}$.
  and its symmetric version. Then, they are proved simultaneously by induction
    on derivations. Similarly for the others.
\end{proof}

% Preservation

Thanks to above lemmas, we can prove Type Preservation. However, we cannot use
induction on the statement of Type Preservation. We prove it as a corollary of
stronger lemma: Lemma~\ref{lemma:EquivalencePreservation}.

\begin{lemma}[Equivalence Preservation]
    \label{lemma:EquivalencePreservation}
    \begin{itemize}
        \item If \( \G \V M : \tau @ A \) and \( M \longrightarrow M' \), then there is \( \tau' \) such that \( \G \V M \E M' : \tau' @ A \) and \( \G \V \tau \E \tau' :: * @ A\).
        \item If \( \G \V \tau :: K @ A \) and \( \tau \longrightarrow \tau' \), then there is \( K' \) such that \( \G \V \tau \E \tau' :: K' @ A \) and \( \G \V K \E K' @ A\).
    \end{itemize}
\end{lemma}

Now, we can show the first important property of \LMD.

\begin{theorem}[Type Preservation]
    \label{theorem:TypePreservation}
    If \( \G \V M : \tau @ A \) and \( M \longrightarrow M' \), then \( \G \V M' : \tau @ A \).
\end{theorem}

\begin{proof}
    From the first item of Lemma \ref{lemma:EquivalencePreservation} and Lemma \ref{lemma:Agreement} and \TConv.
\end{proof}

% Strong Normalization and how to prove it

Strong Normalization is also an important property, which guarantees that no
typed term has an infinite reduction sequence. To prove this, we translate \LMD
to LF~\cite{HarperHonsellPlotkin1993Framework} preserving \( \beta
\)-reductions. Using this translation, we can prove Strong Normalization by
contradiction. If there is an infinite reduction sequence of \LMD, we can map
it to an infinite reduction sequence of LF which contradicts to Strong
Normalization of LF.

\begin{definition}[$\natural$ translation]\
    The $\natural$ translation is a translation from $\lambda^\text{MD}$ to \text{LF}.
    \begin{multicols}{2}
        \begin{itemize}
            \item Kind
                \begin{align*}
                    \natural(*) &= \operatorname{Type} & \\
                    \natural(\Pi x:\tau.K) &= \Pi x:\natural(\tau).\natural(K) &
                \end{align*}
            \item Term
                \begin{align*}
                    \natural(c) &= c & \\
                    \natural(x) &= x & \\
                    \natural(\lambda x:\tau.M) &= \lambda x:\natural(\tau).\natural(M) & \\
                    \natural(M\ N) &= \natural(M)\ \natural(N)& \\
                    \natural(\TB_\alpha M) &= \natural(M) & \\
                    \natural(\TBL_\alpha M) &= \natural(M)& \\
                    \natural(\Lambda\alpha.M) &= \natural(M)& \\
                    \natural(M\ B) &= \natural(M) &
                \end{align*}
            \item Type
            \begin{align*}
                \natural(X) &= X & \\
                \natural(\Pi x:\tau.\sigma) &= \Pi x:(\tau). \natural(\sigma) & \\
                \natural(\tau\ M) &= \natural(\tau) \natural(M) & \\
                \natural(\TW_\alpha \tau) &= \natural(\tau) & \\
                \natural(\forall \alpha.\tau) &= \natural(\tau) &
            \end{align*}
        \item Context
            \begin{align*}
                \natural(\phi) &= \phi & \\
                \natural(\G, x:\tau@A) &= \natural(\G), \natural(x):\natural(\tau) & \\
            \end{align*}
        \item Signature
            \begin{align*}
                \natural(\phi) &= \phi & \\
                \natural(\Sigma, c:\tau) &= \natural(\Sigma),\natural(c):\natural(\tau) & \\
                \natural(\Sigma, X:K) &= \natural(\Sigma), \natural(X):\natural(K) &
            \end{align*}
    \end{itemize}
    \end{multicols}
\end{definition}

Next, we show \( \natural \) translation preserves typing and equivalence of \LMD.

\begin{lemma}[Substitution and $\natural$]\
    \label{lemma:SubstitutionAndNatural}
    If \( \G, x:\sigma \D \V M:\tau@A \) and \( \G \V N:\sigma@A\) in \LMD
    then $\natural(M[x \mapsto N])$ = $\natural(M)[x\mapsto\natural(N)]$
\end{lemma}

\begin{proof}
    By induction on the type derivation tree of $\G, x:\sigma \D \V M:\tau@A$.
\end{proof}

Following Lemma~\ref{lemma:PreservationOfJudgementsInNatural},
\ref{lemma:PreservationOfBetaReductionInNatural},
\ref{lemma:PreservationOfLambdaReductionInNatural},
\ref{lemma:PreservationOfBlacklozengeReductionInNatural} ensure that an
infinite reduction sequence of LF exists if there is an infinite reduction
sequence of \LMD.

\begin{lemma}[Preservation of Judgements in $\natural$]\
    \label{lemma:PreservationOfJudgementsInNatural}
    There are following relations between judgments in \LMD and \text{LF}.
    \begin{itemize}
        \item If \( \vdash \Sigma \) then \( \natural(\Sigma)\ \operatorname{sig} \).
        \item If \( \V \Gamma \) then \( \vdash_{\natural(\Sigma)} \natural(\Gamma) \).
        \item If \( \G \V K \iskind@A \) then \( \natural(\G) \vdash_{\natural(\Sigma)} \natural(K) \).
        \item If \( \G \V \tau : K @ A \) then \( \natural(\G) \vdash_{\natural(\Sigma)} \natural(\tau) : \natural(K) \).
        \item If \( \G \V M : \tau @ A \) then \( \natural(\G) \vdash_{\natural(\Sigma)} \natural(M) : \natural(\tau) \).
        \item If \( \G \V K \E K' @ A \) then \( \natural(\G) \vdash_{\natural(\Sigma)} \natural(K) \E \natural(K') \), \( \natural(\G) \vdash_{\natural(\Sigma)} \natural(K) \) and \( \natural(\G) \vdash_{\natural(\Sigma)} \natural(K') \).
        \item If \( \G \V \tau \E \tau' : K @ A \) then \( \natural(\G) \vdash_{\natural(\Sigma)} \natural(\tau) \E \natural(\tau') \), \( \natural(\G) \vdash_{\natural(\Sigma)} \natural(\tau) : \natural(K) \) and \( \natural(\G) \vdash_{\natural(\Sigma)} \natural(\tau') : \natural(K) \).
        \item If \({ \G \V M \E M' : \tau @ A }\) then \({ \natural(\G) \vdash_{\natural(\Sigma)} \natural(M) \E \natural(M') }\), \({ \natural(\G) \vdash_{\natural(\Sigma)} \natural(M) : \natural(\tau) }\) and \({ \natural(\G) \vdash_{\natural(\Sigma)} \natural(M') : \natural(\tau) }\).
    \end{itemize}
\end{lemma}

\begin{proof}
    We can prove using induction on the type derivation tree.
    We show the case of \TApp{} and \TConv{} as examples.
    Other cases are easy.
    \begin{rneqncase}{\TApp{}}{
            \G \V M : (\Pi(x:\sigma).\tau) @ A \text{ and } \G \V N :\sigma @A
        }
        From the induction hypothesis, we have \({ \natural(\G)
        \vdash_{\natural(\sigma)} \natural(M) : \Pi
        x:\natural(\sigma).\natural(\tau) }\) and \({ \natural(\G)
        \vdash_{\natural(\sigma)} \natural(N) : \natural(\sigma) }\).  Using
        \textsc{B-APP-OBJ} rule in LF, we get \\ \({ \natural(\G)
        \vdash_{\natural(\sigma)} \natural(M)\ \natural(N) : \natural(\tau)[x
        \mapsto \natural(N)] }\).  Because \({ \natural(M)\ \natural(N) =
        \natural(M\ N) }\) from the definition of $\natural$ and \({ \natural(\tau)[x \mapsto \natural(N)] = \natural(\tau[x \mapsto N]) }\) from Lemma~\ref{lemma:SubstitutionAndNatural}, \\
        ${ \natural(\G) \vdash_{\natural(\sigma)} \natural(M\ N) :
        \natural(\tau[x \mapsto N]) }$ in LF.
    \end{rneqncase}
    \begin{rneqncase}{\TConv{}}{
            \G\V M:\tau@A \text{ and } \G\V \tau \E \tau' : K @A
        }
        By the induction hypothesis, \( \natural(\G) \vdash_{\natural(\Sigma)}
        \natural(M):\natural(\tau) \), \( \natural(\G)
        \vdash_{\natural(\Sigma)} \natural(\tau) \E \natural(\tau') \), \(
        \natural(\G) \vdash_{\natural(\Sigma)} \natural(\tau) : \natural(K) \)
        and \( \natural(\G) \vdash_{\natural(\Sigma)} \natural(\tau') :
        \natural(K) \). From Lemma~\ref{lemma:Agreement}, \( K \) is \( * \).
        Therefore, \( \natural(\G) \vdash_{\natural(\Sigma)} \natural(\tau') :
        \text{Type} \). Now, we can use \textsc{B-CONV-OBJ} and get \(
        \natural(\G) \vdash_{\natural(\Sigma)} \natural(M) : \natural(\tau')
        \).
    \end{rneqncase}
\end{proof}

\begin{lemma}[Preservation of $\beta$-Reduction in $\natural$]
    \label{lemma:PreservationOfBetaReductionInNatural}
    If $\G \V M:\tau@A$ and $M \longrightarrow_\beta N$ in $\lambda^{\text{MD}}$
    then $\natural(M) \longrightarrow_\beta^+ \natural(N)$.
\end{lemma}

\begin{proof}
    By induction on the derivation of $\beta$-reduction of \LMD.
    We show only the main case.
    \newcommand{\R}{\longrightarrow_{\beta}}
    \begin{rneqncase}{$(\lambda x:\tau.M)\ N \R M[x \mapsto N]$}{}
        From the definition of $\natural$, $\natural((\lambda x:\tau.M)\ N)$ =
        $\lambda x:\natural(\tau).\natural(M)\ \natural(N)$.  Because $\lambda
        x:\natural(\tau).\natural(M)\ \natural(N)$ is a typed term in LF by
        Lemma~\ref{lemma:PreservationOfJudgementsInNatural}, we can perform
        $\beta$-reduction from it.  As a result of the reduction, we get
        $\natural(M)[x\mapsto\natural(N)]$.  From
        \ref{lemma:SubstitutionAndNatural}, $\natural(M[x \mapsto N])$ =
        $\natural(M)[x\mapsto\natural(N)]$.
    \end{rneqncase}
\end{proof}

\begin{lemma}[Preservation of $\Lambda$-Reduction in $\natural$]
    \label{lemma:PreservationOfLambdaReductionInNatural}
    If $\G \V M:\tau@A$ and $M \longrightarrow_\Lambda N$ in $\lambda^{\text{MD}}$
    then $\natural(M)$ =  $\natural(N)$.
\end{lemma}

\begin{proof}
    We prove by induction on the derivation of $\Lambda$-reduction of \LMD.
    We show only the main case.
    \begin{rneqncase}{\( (\Lambda\alpha.M)\ A \longrightarrow_\Lambda M[\alpha\mapsto A] \)}{}
            By the definition of $\natural$, \(\natural((\Lambda\alpha.M)\ A) =
            \natural(M)\).  Because \(\natural(M)\) does not contain
            \(\alpha\), \(\natural(M[\alpha\mapsto A]) = \natural(M)\).
    \end{rneqncase}
\end{proof}

\begin{lemma}[Preservation of $\blacklozenge$-Reduction in $\natural$]
    \label{lemma:PreservationOfBlacklozengeReductionInNatural}
    If $\G \V M:\tau@A$ and ${M \longrightarrow_\blacklozenge N}$ in $\lambda^{\text{MD}}$
    then $\natural(M)$ =  $\natural(N)$.
\end{lemma}

\begin{proof}
    We prove by induction on the derivation of $\blacklozenge$-reduction of
    \LMD.  We show only the main case.
    \begin{rneqncase}{ \( \TBL_\alpha \TB_\alpha M \longrightarrow_\blacklozenge M \) }{}
        By the definition of $\natural$, \(\natural(\TBL_\alpha \TB_\alpha M) =
        \natural(M)\).
    \end{rneqncase}
\end{proof}

To prove Strong Normalization, we prove that there is no infinite sequence composed of \( \Lambda \) and \( \blacklozenge \)-reductions, first.

\begin{lemma}[Strong Normalization without \( \beta \)-reduction]
    \label{lemma:StrongNormalizationWithoutBetaReduction}
    If \( \G\V M_1:\tau@A \) then there is no infinite sequence of terms $\{M_i\}_{i\ge1}$ which satisfies $M_i \longrightarrow_\Lambda M_{i+1}$ or $M_i \longrightarrow_\blacklozenge M_{i+1}$ for $i\ge 1$.
\end{lemma}

\begin{proof}
    % Prove by lexicographical order.
    We write the number of \( \Lambda \) in term \( M \) as \( \#\Lambda_M \) and the number of \( \TB \) in term \( M \) as \( \#\TB_M \). For \( i \ge 1 \), the pair \( ( \sharp \Lambda_{M_i}, \sharp \TB_{M_i} ) \) is strictly greater than the pair \( (\#\Lambda_{M_{i+1}}, \#\TB_{M_{i+1}}) \) in lexicographical order because \( \Lambda \)-reduction reduces the number of \( \Lambda \) by one and \( \blacklozenge \)-reduction reduces the number of \( \TB \) by one preserving the number of \( \Lambda \). Now, the lemma follows from well-foundedness of pairs of natural number.

    % Prove not by lexicographical order
    % Prove by contradiction. If there is a such infinite sequence \( \{M_i\}_{i\ge1} \), there is another infinite sequence \( \{ N_i \}_{ i \ge 1 } \) which satisfies $N_i \longrightarrow_\blacklozenge N_{i+1}$ for \( i \ge 1 \). This is because both \( \Lambda \) and \( \blacklozenge \)-reductions reduce or preserve the number of \( \Lambda \) in the term, therefore we can perform \( \Lambda \)-reduction only a finite number of times. However, \( \blacklozenge \)-reduction always reduces the size of the term, thus we can perform \( \blacklozenge \)-reduction only finite a finite number of times also. 
\end{proof}

Then, we can prove Strong Normalization of \LMD.

\begin{theorem}[Strong Normalization]
    \label{theorem:StrongNormalization}
    If \( \G\V M_1:\tau@A \) then there is no infinite sequence of terms $\{M_i\}_{i\ge1}$ which satisfies $M_i \longrightarrow M_{i+1}$ for $i\ge 1$.
\end{theorem}

\begin{proof}

    If there is an infinite reduction sequence in \LMD then there are infinite \( \beta \)-reductions in the sequence. If there are only finite \( \beta \)-reductions in the sequence, we can construct another infinite reduction sequence which is composed only of \( \Lambda \)-reduction and \( \blacklozenge \)-reduction. However, it contradicts Lemma~\ref{lemma:StrongNormalizationWithoutBetaReduction}.
    
    Then, we show Strong Normalization by proof by contradiction. 
    
    We assume that there is an infinite reduction sequence \( \{ M_i \} \) of \LMD and \( \G\V M_1:\tau@A \). Thanks to discussion above, we can assume it contains infinite \( \beta \)-reductions. From \( \{ M_i \} \), we can construct another infinite reduction sequence \( \{ \natural(M_i) \} \). From Lemma~\ref{lemma:PreservationOfBetaReductionInNatural}, \ref{lemma:PreservationOfLambdaReductionInNatural}, \ref{lemma:PreservationOfBlacklozengeReductionInNatural}, there are infinite \( \beta \)-reductions in \( \{ \natural(M_i) \} \).
    
    However, \( \natural(\G) \vdash_{\natural(\Sigma)} \natural(M_1):\natural(\tau) \) from Lemma~\ref{lemma:PreservationOfJudgementsInNatural} therefore the existence of \( \{ \natural(M_i) \} \) contradicts Strong Normalization of LF.
\end{proof}

Confluence is a property that any reduction sequences from one typed term
converge.  Since we have proved Strong Normalization, we can use Newman's
Lemma~\cite{BaaderTobias1998TermRewriting} to prove Confluence.

\begin{theorem}[Confluence]
    \label{theorem:confluence}
	For any term $M$, if $M \longrightarrow^* M'$ and $M \longrightarrow^* M''$ then
	there exists $M'''$ that satisfies $M' \longrightarrow^* M'''$ and $M'' \longrightarrow^* M'''$.
\end{theorem}

\begin{proof}
  We can easily show Weak Church-Rosser. Use Newman's Lemma.
	% Because we proved Strong Normalization of \LMD, 
	% we can use Newman's lemma to prove Confluence of \LMD.
	% Then, what we must show is Weak Church-Rosser Property now.
	% When we consider two different redexes in a \LMD term, they can only be disjoint, or one is a part of the other.
	% In short, they are never overlapped each other.
	% So, we can reduce one of them after we reduce another.
\end{proof}

Now, we turn our attention to staged semantics.  First, the staged
reduction relation is a subrelation of full reduction, so Subject
Reduction holds also for the staged reduction.

\begin{lemma}[Staged Reduction and Normal Reduction]
  If $M \longrightarrow_s M'$, then $M \longrightarrow M'$.
\end{lemma}
\begin{proof}
    Easy from the definition of \( M \longrightarrow_s M' \).
\end{proof}

The following theorem Unique Decomposition ensures that every typed
term is either a value or can be uniquely decomposed to an evaluation
context and a redex, ensuring that a well-typed term is not
immediately stuck and the staged semantics is deterministic.

\begin{theorem}[Unique Decomposition]
  If $\G$ does not have any variable declared at stage $\varepsilon$ 
  and $\G \V M : \tau @ A$, then either
  \begin{enumerate}
  \item $ M \in V^A$, or
  \item $M$ can be uniquely decomposed into an evaluation context and a redex, that is, there uniquely exist $B, E^A_B$, and $R^B$ such that $M = E^A_B[R^B]$.
  \end{enumerate}
\end{theorem}

\begin{proof}
  By straightforward induction on typing derivations.
\end{proof}

The type environment $\G$ of a statement usually must be empty; in other
words, the term must be closed. The condition is relaxed here because
variables at stages higher than \(\varepsilon\) are considered symbols. In
fact, this relaxation is required for proof by induction to work.

Progress is a corollary of Unique Decomposition.

\begin{theorem}[Progress]
	If $\G$ does not have any variable declared at stage $\varepsilon$ and $\G \V M : \tau  @ A$, then
	$ M \in V^A $ or there exists $M'$ such that $M \longrightarrow_s M'$.
\end{theorem}

\subsection{Properties of Algorithmic Typing}

In this subsection, we prove the properties of algorithmic typing of \LMD. The
main purpose of this subsection is proving the completeness and soundness of
algorithmic typing, which gives the equivalence with the original typing. We
start by proving the properties of algorithmic reduction because it is used to
define term equivalence in \QAANF. \( \ANF(M) \) is the normal form of a term
\( M \) in the algorithmic reduction and \( \E_\alpha \) means \( \alpha \)
equivalence.

\begin{center}
    \infrule[\QAANF]{
        \ANF(M) \E_\alpha \ANF(M') \andalso
        \G \AVS M : \tau @ A \andalso
        \G \AVS M' : \tau' @ A
    }{
        \G \AVS M \E M' @A
    }
\end{center}

First, we prove the uniqueness of \( \ANF(M) \) for all typed term \( M \)
using Strong Normalization and Confluence of algorithmic reduction.
Fortunately, we can prove these two properties similar to the original
reduction.

\begin{lemma}[Strong Normalization of Algorithmic Reduction]
    \label{lemma:StrongNormalizationofAlgorithmicReduction}
    If \( \G \AVS M :\tau @ A\), there is no infinite sequence of \( \RA \) reductions from \( M \).
\end{lemma}

\begin{proof}
    Proved by using \( \natural \) function.
\end{proof}

\begin{lemma}[Confluence of Algorithmic Reduction]
    \label{lemma:ConfluenceofAlgorithmicReduction}
    If \({ \G \AVS M :\tau @ A }\), \\ \({ M \RA^* M' }\) and \({ M \RA^* M'' }\)
    then there is a \( M''' \) such that \({ M' \RA^* M''' }\) and \({ M'' \RA^* M''' }\).
\end{lemma}

\begin{proof}
    By argument similar that in Thorem \ref{theorem:confluence} 
\end{proof}

\begin{lemma}[Uniqueness of Algorithmic Reduction Normal Form]
    \label{lemma:UniquenessOfANF}
    If \\ \({ \G \AVS M : \tau @ A }\) then there is the unique normal form of \( M \), \( \ANF(M) \).
\end{lemma}

\begin{proof}
    Immediate from Lemma \ref{lemma:StrongNormalizationofAlgorithmicReduction} and \ref{lemma:ConfluenceofAlgorithmicReduction}.
\end{proof}

Then, we prove the relationship between one step algorithmic reduction and substitutions.

\begin{lemma}[Algorithmic Reduction and Term Substitution]
    \label{lemma:AlgorithmicReductionAndTermSubstitution}
    If \\ \( \G, z:\xi@B, \D \AVS M : \tau @ A \), \( \G \AVS P : \xi @ B \) and \( M \RA M' \) then \\
    \( \ANF(M[z\mapsto P]) \E_\alpha \ANF(M'[z\mapsto P]) \).
\end{lemma}

\begin{proof}
    By induction on the derivation of \( M \RA M' \).
    \begin{rneqncase}{\textsc{A-\%}}{\%M \RA M \text{ and } \text{FV}(M)=\emptyset}
        From \( \text{FV}(M)=\emptyset \), \( z \notin \text{FV}(M) \) and \( z \notin \text{FV}(M') \).
        Then \( \ANF(M) \E_\alpha \ANF(M') \) is clear.
    \end{rneqncase}

    \begin{rneqncase}{\textsc{A-$\beta$}}{(\lambda x:\tau.M)\ N \RA M[x\mapsto N]}
        \begin{align*}
            & \ANF(((\lambda x:\tau.M)\ N)[z \mapsto P]) \\
            & = \ANF((\lambda x:\tau.M[z \mapsto P])\ N[z \mapsto P]) \\
            & = \ANF(M[z \mapsto P][x \mapsto N[z \mapsto P]]) & (\text{From Lemma } \ref{lemma:UniquenessOfANF}) \\
            & = \ANF(M[x \mapsto N][z \mapsto P]) \\
            & = \ANF(M'[z \mapsto P])
        \end{align*}
    \end{rneqncase}
\end{proof}

\begin{lemma}[Algorithmic Reduction and Stage Substitution]
    \label{lemma:AlgorithmicReductionAndStageSubstitution}
    If \( \G \AVS M : \tau @ A \) and \( M \RA M' \) then
    \( \ANF(M[\alpha \mapsto B]) \E_\alpha \ANF(M'[\alpha \mapsto B]) \).
\end{lemma}

\begin{proof}
    By induction on the derivation of \( M \RA M' \).
\end{proof}

Next, we prove that we can derive the same term in two different ways. On the
one hand, we may calculate the algorithmic normal form after the substitution.
On the other hand, we may calculate the algorithmic normal form after the
substitution for the algorithmic normal form. This property (state formally in
Lemma \ref{lemma:AlgorithmicNomalFormAndTermSubstitution}) is crucial for Term
Substitution Lemma of Algorithmic Judgement and Stage Substitution Lemma of
Algorithmic Judgement (Lemma
\ref{lemma:TermSubstitutionLemmaOfAlgorithmicJudgement} and Lemma
\ref{lemma:StageSubstitutionLemmaofAlgorithmicJudgement}). In \cite{benjamin2005attapldependent},
they use weak head normal form, but here we use normal form because we couldn't
prove this lemma for weak head normal form.

\begin{lemma}[Algorithmic Normal Form and Term Substitution]
    \label{lemma:AlgorithmicNomalFormAndTermSubstitution}
    If \\ \( \G, z:\xi@B, \D \AVS M : \tau @ A \) and \( \G \AVS P : \xi @ B \) then \\
    \({ \ANF(M[z\mapsto P]) \E_\alpha \ANF(\ANF(M)[z\mapsto P]) }\).
\end{lemma}

\begin{proof}
    Prove by induction on the length \( n \) of reduction from \( M \) to \( \ANF(M) \).
    \begin{rneqncase}{$n$ = 0}{M = \ANF(M) }
        It is immediate.
    \end{rneqncase}
    \begin{rneqncase}{$n$ = 1}{M \RA M' \text{ and } \ANF(M) = M' }
        From Lemma \ref{lemma:AlgorithmicReductionAndTermSubstitution}.
    \end{rneqncase}
    \begin{rneqncase}{$n  > 1$}{M \RA^* M' \text{ and } M' \RA^* M'' \text{ and } \ANF(M) = M'' }
        From the induction hypothesis, \( \ANF(M[z\mapsto P]) \E_\alpha \ANF(M'[z\mapsto P]) \).
        From Lemma \ref{lemma:AlgorithmicReductionAndTermSubstitution}, \( \ANF(M'[z\mapsto P]) \E_\alpha \ANF(M''[z\mapsto P]) \).
        Then from the transitivity of \( \E_\alpha \), \( \ANF(M[z\mapsto P]) \E_\alpha \ANF(\ANF(M)[z\mapsto P]) \).
    \end{rneqncase}
\end{proof}

\begin{lemma}[Algorithmic Normal Form and Stage Substitution]
    \label{lemma:AlgorithmicNomalFormAndStageSubstitution}
    If \\ \( \G \AVS M : \tau @ A \) then
    \( \ANF(M[\alpha \mapsto B]) \E_\alpha \ANF(\ANF(M)[\alpha \mapsto A]) \)
\end{lemma}

\begin{proof}
    Same as Lemma \ref{lemma:AlgorithmicNomalFormAndTermSubstitution} using
    Lemma \ref{lemma:AlgorithmicReductionAndStageSubstitution} instead of Lemma
    \ref{lemma:AlgorithmicReductionAndTermSubstitution}.
\end{proof}

Thanks to Lemma \ref{lemma:AlgorithmicNomalFormAndTermSubstitution} and Lemma
\ref{lemma:AlgorithmicNomalFormAndStageSubstitution}, we can prove two
substitution lemmas.

\begin{lemma}[Term Substitution Lemma of Algorithmic Judgement]
    \label{lemma:TermSubstitutionLemmaOfAlgorithmicJudgement}
    If $\G, z:\xi@B, \D \AVS \mathcal{J}$ and $\G\AVS P:\xi @B$ then $\G, \D[z \mapsto P] \AVS \mathcal{J}[z \mapsto P]$.
\end{lemma}

\begin{proof}
    By induction on the derivation of \( \G, z:\xi@B, \D \AVS \mathcal{J}
    \). Most cases are the same with Lemma \ref{lemma:TermSubstitution}. We show
    only different cases.

    \begin{rneqncase}{\QAANF}{
            \G, z:\xi @ B, \D \AVS M \E N @ A \\
            \text{ is derived from }
            \ANF(M) \E_\alpha \ANF(N).
        }
    \end{rneqncase}
    \( \ANF(\ANF(M)[z \mapsto P]) \E_\alpha \ANF(\ANF(N)[z \mapsto P]) \) is obvious from \( \ANF(M) \E_\alpha \ANF(N) \).
    From Lemma \ref{lemma:AlgorithmicNomalFormAndTermSubstitution}, \( \ANF(M[z \mapsto P]) \E_\alpha \ANF(N[z \mapsto P]) \).
    Then \( \G, \D[z \mapsto P] \AVS M[z \mapsto P] \E N[z \mapsto P] @ A \).
\end{proof}

\begin{lemma}[Stage Substitution Lemma of Algorithmic Judgement]
    \label{lemma:StageSubstitutionLemmaofAlgorithmicJudgement}
    If $\G \AVS \mathcal{J}$ then $\G[\alpha \mapsto B] \AVS \mathcal{J}[\alpha \mapsto B]$.
\end{lemma}

\begin{proof}
    Same as Lemma \ref{lemma:TermSubstitutionLemmaOfAlgorithmicJudgement} using
    Lemma \ref{lemma:AlgorithmicNomalFormAndStageSubstitution} instead of Lemma
    \ref{lemma:AlgorithmicNomalFormAndTermSubstitution}.
\end{proof}

\begin{lemma}[Agreement of Algorithmic Typing]
    Following statements are valid simultaneously.
    \label{lemma:AgreementofAlgorithmicTyping}
    \begin{itemize}
        \item If \( \G \AVS \tau :: K @ A \) then \( \G \AVS K \iskind@ A \).
        \item If \( \G \AVS M : \tau @ A \) then \( \G \AVS \tau :: * @ A \).
    \end{itemize}
\end{lemma}

\begin{lemma}[Symmetricity in Algorithmic Equivalence]
    The algorithmic equivalence relationship is symmetrical.
    \label{lemma:SymmetricityinAlgorithmicEquivalence}
    \begin{itemize}
        \item If \( \G \AVS K \E K' @ A \) then \( \G \AVS K' \E K @ A \).
        \item If \( \G \AVS \tau \E \tau' @ A \) then \( \G \AVS \tau' \E \tau @ A \).
        \item If \( \G \AVS M \E M' @ A \) then \( \G \AVS M' \E M @ A \).
    \end{itemize}
\end{lemma}

\begin{proof}
    It is obvious because all algorithmic equivalence rules are symmetrical.
\end{proof}

\begin{lemma}[Reflexivity in Algorithmic Equivalence]
    The algorithmic equivalence relationship is reflexive.
    \label{lemma:ReflexivityinAlgorithmicEquivalence}
    \begin{itemize}
        \item If \( \G \AVS K \E K @ A \) then \( \G \AVS K \E K @ A \).
        \item If \( \G \AVS \tau \E \tau @ A \) then \( \G \AVS \tau \E \tau @ A \).
        \item If \( \G \AVS M \E M @ A \) then \( \G \AVS M \E M @ A \).
    \end{itemize}
\end{lemma}

\begin{proof}
    Prove by induction.
\end{proof}

\begin{lemma}[Transition in Algorithmic Equivalence]
    The algorithmic equivalence relationship is transitive.
    \label{lemma:TransitioninAlgorithmicEquivalence}
    \begin{itemize}
        \item If \( \G \AVS K \E K' @ A \) and \( \G \AVS K' \E K'' @ A \) then \( \G \AVS K \E K'' @ A \).
        \item If \( \G \AVS \tau \E \tau' @ A \) and \( \G \AVS \tau' \E \tau'' @ A \) then \( \G \AVS \tau \E \tau'' @ A \).
        \item If \( \G \AVS M \E M' @ A \) and \( \G \AVS M' \E M'' @ A \) then \( \G \AVS M \E M'' @ A \).
    \end{itemize}
\end{lemma}

\begin{proof}
    Prove by the induction on derivations.  In this lemma, there are two
    hypothesis \( \mathcal{J}_1 \) and \( \mathcal{J}_2 \).  The last rules of
    \( \mathcal{J}_1 \) and \( \mathcal{J}_2 \) are the same.  Because these
    are algorithmic derivation so the last rule of derivation is decided
    uniquely by the shape of judgments.  Then we can prove this theorem easily
    from the induction hypothesis and the last rule.
\end{proof}

\begin{lemma}[Equivalence Preservation of Algorithmic Reduction]
    \label{lemma:EquivalencePreservationOfAlgorithmicReduction}
    This is an algorithmic version of Lemma~\ref{lemma:EquivalencePreservation}.
    \begin{itemize}
        \item If \( \G \V M : \tau @ A \) and \( M \RA M' \) then \( \G \V M \E M' : \tau @ A \).
        \item If \( \G \V \tau :: K @ A \) and \( \tau \RA \tau' \), then \( \G \V \tau \E \tau' :: K \).
    \end{itemize}
\end{lemma}
\begin{proof}
    By induction on the type or kind derivation tree. We show the different case from Lemma~\ref{lemma:EquivalencePreservation}.

    \begin{rneqncase}{\TCsp}{
            \G\V \%_\alpha M:\tau@{A\alpha} \text{ is derived from }
            \G\V M:\tau@A \text{ and }\G\V \tau::*@{A\alpha}.
        }
        \begin{itemize}
                \item \textsc{A-\%} \\
                    \( \%_\alpha M \RA M \) when \( \text{FV}(\%_\alpha M) =
                    \emptyset \). In this case \( M \) is a closed term
                    therefore contains no free variables. So, we can derive \(
                    \G \V M : \tau @ A\alpha \) and \( \G \V \%_\alpha M \E M @
                    A\alpha \) from \QPercent.
        \end{itemize}
    \end{rneqncase}
\end{proof}

\begin{lemma}[Type Preservation of Algorithmic Reduction]
    \label{lemma:TypePreservationofAlgorithmicReduction}
    If \( \G \V M:\tau @A \) and \( M \RA M' \) then \( \G \V M' : \tau @ A \).
\end{lemma}

\begin{proof}
    Similar to Theorem \ref{theorem:TypePreservation}, we prove type and
    equivalence preservation lemma and use them. The different case is
    \textsc{A-\%}.  If \( \G \V \%_\alpha M : \tau @ A \), \( \%_\alpha M \RA M
    \) and \( \text{FV}(\%_\alpha M) = \emptyset \), \( M \) is a closed term.
    So \( \G \V M : \tau @ A \).
\end{proof}

Now, we can say term \( M \) and \( \ANF(M) \) are equivalent.

\begin{lemma}[ANF and Term Equivalence]
    \label{lemma:ANFandTermEquivalence}
    If \( \G \V M : \tau @ A \) then \({ \G \V M \E \ANF(M) : \tau @ A }\).
\end{lemma}
\begin{proof}
    Corollary of Lemma \ref{lemma:TypePreservationofAlgorithmicReduction} and
    \ref{lemma:EquivalencePreservationOfAlgorithmicReduction}.
\end{proof}

We prove a seemingly trivial lemma here. It plays an important role in \QAANF
case of the Soundness proof.

\begin{lemma}[Types of the Same Term]
    \label{lemma:TypesOfTheSameTerm}
    If \( \G \V M : \tau @ A \) and \( \G \V M : \tau' @ A \) then \( \G \V \tau \E \tau' :: * @ A \).
\end{lemma}

\begin{proof}
    Prove by induction on \( \G \V M : \tau @ A \).
    \begin{rneqncase}{\TConv}{
            \G \V M : \tau @ A \text{ is derived from }
            \G \V M : \tau'' @ A \text{ and } \G \V \tau \E \tau'' :: * @ A.
        }
        From the induction hypothesis, \( \G \V \tau \E \tau'' :: * @ A \). From \QTrans, \( \G \V \tau \E \tau' :: * @ A \).
    \end{rneqncase}
    \begin{rneqncase}{\TAbs}{
            \G \V (\lambda (x:\sigma).M) : (\Pi (x:\sigma).\tau)@A \text{ is derived from }
            \G\V \sigma::*@A \text{ and }
            \G,x:\sigma@A\V M:\tau@A.
        }
        Use Lemma~\ref{lemma:Inversion} on \( \G \V (\lambda
        (x:\sigma).M) : \tau' @ A \), there are \( \sigma'' \) and \( \tau'' \)
        such that \( \tau' = \Pi x:\sigma''.\tau'' \), \( \G \V \sigma \E
        \sigma'' @ A \) and \( \G, x:\sigma''@A \V M : \tau'' @ A \).
        Therefore, \( \G \V \Pi x:\sigma.\tau \E \Pi x:\sigma''.\tau'' @ A \)
        from \QTAbs.
    \end{rneqncase}
    \begin{rneqncase}{\TVar}{
            \G \V x:\tau@A \text{ is derived from }
            x:\tau@A \in \G.
        }
        From the well-formedness of \( \G \), there is no duplicate entry in \(
        \G \). Therefore, \( \tau = \tau' \) or \( \G \V \tau \E \tau' @ A \)
        by \TConv.
    \end{rneqncase}
\end{proof}


Finally, we prove the soundness of algorithmic typing. It means that if a term
is typed in algorithmic typing then the term is also typed in the original
typing.  Because algorithmic typing is defined mutually involving other
algorithmic judgments, we use simultaneous induction for proof.

\begin{theorem}[Soundness of Algorithmic Typing]
    \label{theorem:SoundnessOfAlgorithmicTyping}
    If a term is typed in algorithmic rules then the term is typed in normal rules, too.
    \begin{itemize}
        \item If \(\G\AVS K \iskind @ A \) then \(\G\V K \iskind @A \).
        \item If \(\G\AVS \tau :: K @ A \) then \(\G\V \tau ::K  @ A \).
        \item If \(\G\AVS M:\tau @ A \) then \(\G\V M:\tau @ A \).
        \item If \(\G\AVS K,K' @ A\) and \(\G\AVS K\E K' @ A \) then \(\G\V K\E K' @ A \).
        \item If \( \G \AVS \tau :: K @ A \), \( \G \AVS \tau' :: K' @ A \), \( \G \AVS K \E K' @ A\) and \(\G\AVS \tau \E \tau' @ A \) then \(\G\V \tau \E \tau' :: K @ A \).
        \item If \(\G \AVS M : \tau @ A\), \( \G \AVS M' : \tau' @ A \), \( \G \AVS \tau \E \tau' @ A\) and \(\G\AVS M\E M' @ A \) then \(\G\V M\E M' :: \tau @ A \).
    \end{itemize}
\end{theorem}

\begin{proof}
    We can prove using induction on the derivation tree.
    \begin{rneqncase}{\KAApp{}}{
            \G\AVS \sigma\ M::K[x\mapsto M]@A \\
            \text{ is derived from } 
            \G\AVS \sigma:: (\Pi x:\tau.K)@A \text{ , }
            \G\AVS M:\tau'@A \text{ and } 
            \G\AVS \tau \E \tau' @ A.
        }
        \( \G \AVS \tau::* \) from the derivation of \( \G\AVS \sigma:: (\Pi
        x:\tau.K)@A \), . Using Lemma~\ref{lemma:AgreementofAlgorithmicTyping}
        on \( \G\AVS M:\tau'@A \), we get \( \G \AVS \tau' :: * @ A \). Using
        the induction hypothesis on \( \G \AVS \tau, \tau'::* @ A \) and \(
        \G\AVS \tau \E \tau' @ A \), we get \( \G\V \tau \E \tau' :: * @ A \).
        From the induction hypothesis, \( \G\V \sigma:: (\Pi x:\tau.K)@A \) and
        \( \G\V M:\tau'@A \). Then, \TConv and \KApp guide to \( \G\V \sigma\
        M::K[x\mapsto M]@A \).
    \end{rneqncase}
    \begin{rneqncase}{\TAApp}{
            \G\AVS M\ N : \tau[x\mapsto N]@A \\
            \text{ is derived from } 
            \G\AVS M:(\Pi (x:\sigma).\tau) \text{ , }
            \G\AVS N:\sigma'@A \text{ and }
            % \G\AVS \sigma'::*@A \text{ and }
            \G\AVS \sigma\E\sigma' @A.
        }
        \( \G \AVS \sigma :: * @ A \) from the derivation of \( \G\AVS M: (\Pi
        x:\sigma.\tau)@A \). Using
        Lemma~\ref{lemma:AgreementofAlgorithmicTyping} on \( \G\AVS N:\sigma'@A
        \), we get \( \G \AVS \sigma' :: * @ A \). Just same as \KApp case, \(
        \G\V \sigma\E\sigma' :: * @A \).  From the induction hypothesis, \(
        \G\V M:(\Pi (x:\sigma).\tau) @ A\) and \( \G\V N:\sigma'@A \). Then, \(
        \G\V M\ N : \tau[x\mapsto N]@A \) from \TConv and \TApp.
    \end{rneqncase}
    \begin{rneqncase}{\QAANF}{
            \G \AVS M \E N@A
            \text{ is derived from } \\
            \ANF(M) \E_\alpha \ANF(N),
            \G \AVS M : \tau @ A \text { and }
            \G \AVS N : \tau' @ A.
        }
        From Lemma~\ref{lemma:AgreementofAlgorithmicTyping}, \( \G \AVS \tau ::
        * @ A \) and \( \G \AVS \tau' :: * @ A \).  Using the induction
        hypothesis on \( \G \AVS M : \tau @ A \) and \( \G \AVS N : \tau' @ A
        \), respectively, we get \( \G \V M : \tau @ A \) and \( \G \V N :
        \tau' @ A \). Using Corollary~\ref{lemma:ANFandTermEquivalence} on \(
        \G \V M : \tau @ A \), we get \( \G \V M \E \ANF(M) : \tau @ A \). From
        Lemma~\ref{lemma:Agreement}, \( \G \V M : \tau @ A \) and \( \G \V
        \ANF(M) : \tau @ A \). From
        Corollary~\ref{lemma:ANFandTermEquivalence}, \( \G \V N \E \ANF(N) :
        \tau' @ A \). From Lemma~\ref{lemma:Agreement}, \( \G \V N : \tau' @ A
        \) and \( \G \V \ANF(N) : \tau' @ A \).  Because \( \ANF(M) \E_\alpha
        \ANF(N) \), \( \G \V \ANF(M) : \tau @ A \) and \( \G \V \ANF(N) : \tau'
        @ A \), \( \G \V \tau \E \tau' @ A \) from
        Lemma~\ref{lemma:TypesOfTheSameTerm}.  Then, using \TConv, we get \( \G
        \V N : \tau @ A \). Finally, we can prove \( \G \V M \E N@A \) from
        Corollary~\ref{lemma:ANFandTermEquivalence}, \QTrans and \QSym.
    \end{rneqncase}
\end{proof}

The next theorem is the completeness of algorithmic typing. First, we prove
symmetricity, reflexivity and transitivity of algorithmic typing, which are
indispensable for the proof.


This lemma is used in \QPercent case in the proof of the completeness.

\begin{lemma}[Free Variable and Stage]
    \label{lemma:FreeVariableandStage}
    If \( \G\V M:\tau@{A\alpha} \) and \( \G\V M:\tau @A \) then \( \FV(M) = \emptyset \).
\end{lemma}

\begin{proof}
    Prove by contradiction. Assume \( \G\V M:\tau@{A\alpha} \) and \({ \G\V
    M:\tau @A }\) and \( \FV(M) \neq \emptyset \). Then, \( x : \tau @ B, x:
    \tau @ B\alpha \in \G \). However, from well-formedness of \( \G \), there
    is no such \( x \).
\end{proof}

This is the last theorem -- Completeness. It means that when a term is typed in
the original typing, it is also typed in the algorithmic typing.

\begin{theorem}[Completeness of Algorithmic Typing]\
    \label{theorem:CompletenessofAlgorithmicTyping}
    \begin{itemize}
        \item If \(\G\V K \) then \(\G\AVS K \).
        \item If \(\G\V \tau ::K @ A \) then there is \(K'\) such that \( \G \AVS K' @ A \) and \({ \G \AVS K\E K' @ A }\) and \({ \G\AVS \tau :: K' @ A }\).
        \item If \(\G\V M:\tau @ A \) then there is \(\tau'\) such that \({ \G \AVS \tau' :: * @ A }\) and \({ \G \AVS \tau \E \tau' @ A }\) and \({ \G\AVS M:\tau' @ A }\).
        \item If \(\G\V K\E K' @ A \) then \(\G\AVS K\E K' @ A \).
        \item If \(\G\V \tau \E \tau' :: K @ A \) then \(\G\AVS \tau \E \tau' @ A \).
        \item If \(\G\V M\E M' : \tau @ A \) then \(\G \AVS M \E M' @ A \).
    \end{itemize}
\end{theorem}

\begin{proof}
    Prove by induction on the derivation.
    \begin{rneqncase}{\KApp}{
            \G\V \sigma\ M::K[x\mapsto M]@A \\
            \text{ is derived from } 
            \G\V \sigma:: (\Pi x:\tau.K)@A \text{ and } \G\V M:\tau@A.
        }
        Apply the induction hypothesis on \( \G\V \sigma:: (\Pi x:\tau.K)@A \),
        there is a kind \( J \) such that \( \G\AVS J @A \), \( \G\AVS \sigma
        :: J \) and \( \G\AVS J \E (\Pi x:\tau.K)@A \).  The last rule of the
        derivation of \( \G\AVS J \E (\Pi x:\tau.K)@A \) is \QTAAbs, so \( J =
        \Pi x:\tau'.K' \), \( \G \AVS \tau \E \tau' @ A \) and \( \G, x:\tau
        \AVS K \E K' @ A \).  Use Lemma \ref{lemma:Agreement} and the induction
        hypothesis on \( \G \V M : \tau @ A \), \( \G \AVS \tau :: * @ A \).
        Now, we can apply \KAApp to \( \G \AVS \sigma :: \Pi x:\tau'.K' @ A \),
        \( \G \AVS M: \tau @ A \) and \( \G \AVS \tau \E \tau' @ A\) and get \(
        \G \AVS \sigma\ M :: K'[x \mapsto M] @ A \).  From Lemma
        \ref{lemma:TermSubstitutionLemmaOfAlgorithmicJudgement}, \( \G \AVS K[x
        \mapsto M] \E K'[x \mapsto M] @ A \).
    \end{rneqncase}
    \begin{rneqncase}{\KConv}{
            \G\V \tau::J@A \\
            \text{ is derived from }
            \G\V \tau::K@A \text{ and }
            \G\V K\equiv J@A.
        }
        Apply the induction hypothesis on \( \G\V \tau::K@A \),
        there is a kind K' such that \( \G \AVS K' @ A \) and 
        \( \G \AVS K\E K' @ A \) and \( \G\AVS \tau :: K' @ A \).
        Apply the induction hypothesis on \( \G\V K\equiv J@A \),
        \( \G\AVS K\equiv J@A \).
        From Lemma \ref{lemma:SymmetricityinAlgorithmicEquivalence} and
        \ref {lemma:TransitioninAlgorithmicEquivalence}, \( \G \AVS J \E K' @ A \).
    \end{rneqncase}
    \begin{rneqncase}{\QTApp}{
            \G\V \tau\ M \E \sigma\ N :: K[x \mapsto M]@A 
            \text{ is derived from } \\
            \G\V \tau \E \sigma :: (\Pi x:\rho.K)@A \text{ and }
            \G\V M \E N : \rho @A.
        }
        From the induction hypothesis, \( \G\AVS \tau \E \sigma@A \) and \(
        \G\AVS M \E N @A \). Use \TAApp, \( \G\AVS \tau\ M \E \sigma\ N \).
    \end{rneqncase}
    \begin{rneqncase}{\TConv}{
            \G\V M:\tau'@A \\
            \text{ is derived from }
            \G\V M:\tau@A \text{ and }
            \G\V \tau\equiv \tau':: K@A.
        }
        Apply the induction hypothesis on \( \G\V \tau\equiv \tau':: K@A \), \(
        \G\AVS \tau\equiv \tau' @A \).  Use Lemma \ref{lemma:Agreement} and the
        induction hypothesis to \( \G\V M : \tau @A \), we get \( \G\AVS \tau
        :: * @ A \). Now we can see \( \tau \) as \( \tau' \) in the second
        item of Theorem \ref{theorem:CompletenessofAlgorithmicTyping}.
    \end{rneqncase}
    \begin{rneqncase}{\QPercent}{
            \G\V \%_\alpha M \E M : \tau@{A\alpha}
            \text{ is derived from } \\
            \G\V M:\tau@{A\alpha} \text{ and }
            \G\V M:\tau@A.
        }
        From Lemma~\ref{lemma:FreeVariableandStage}, \( \FV(M) = \emptyset \).
        Then, \( \%_\alpha M \RA M \) and \( \ANF(\%_\alpha M) = \ANF(M) \).
        Using the induction hypothesis on \( \G\V M:\tau@{A\alpha} \), there is
        a \( \tau' \) such that \( \G \AVS \tau' :: * @ A\alpha \), \( \G \AVS
        \tau \E \tau' @ A\alpha \) and \( \G \AVS M : \tau' @ A\alpha \).
        Using the induction hypothesis on \( \G\V M:\tau@A \), there is a \(
        \tau'' \) such that \( \G \AVS \tau'' :: * @ A \), \( \G \AVS \tau \E
        \tau'' @ A \) and \( \G \AVS M : \tau'' @ A \).  Using \TACsp on \( \G
        \AVS M : \tau'' @ A \), we get \( \G \AVS \%_\alpha M : \tau'' @
        A\alpha \).  By using \QAANF on \( \ANF(\%_\alpha M) = \ANF(M) \), \( \G
        \AVS \%_\alpha M : \tau'' @ A\alpha \) and \( \G \AVS M : \tau'' @ A
        \), we get \( \G \AVS \%_\alpha M \E M @ {A\alpha} \)
    \end{rneqncase}
    \begin{rneqncase}{\QBeta}{
            \G\V(\lambda x:\sigma.M)\ N\E M[x\mapsto N] : \tau[x \mapsto N]@A
            \text{ is derived from } \\
            \G,x:\sigma@A\V M:\tau@A \text{ and }
            \G\V N:\sigma@A.
        }
        \( \ANF((\lambda x:\sigma.M)\ N) \E_\alpha \ANF(M[x\mapsto N]) \)
        because \( (\lambda x:\sigma.M)\ N \longrightarrow_A M[x\mapsto N] \).
        Using the induction hypothesis on \( \G,x:\sigma@A\V M:\tau@A \), there
        is \( \tau' \) such that \( \G,x:\sigma@A \AVS \tau' :: * @ A \) and \(
        \G,x:\sigma@A \AVS \tau \E \tau' @ A \) and \( \G,x:\sigma@A\AVS
        M:\tau'@A \). Using the induction hypothesis on \( \G\V N:\sigma@A \),
        there is \( \sigma' \) such that \( \G \AVS \sigma' :: * @ A \), \( \G
        \AVS \sigma \E \sigma' @ A \) and \( \G \AVS N : \sigma' \). From
        \TAApp, \( \G \AVS M\ N : \tau'[x \mapsto N]@A \). From
        Lemma~\ref{lemma:TermSubstitutionLemmaOfAlgorithmicJudgement}, \( \G
        \AVS M[x \mapsto N] : \tau'[x \mapsto N] \). Then, we can use \QAANF.
    \end{rneqncase}
    \begin{rneqncase}{\QKSym}{}
        It is obvious from Lemma \ref{lemma:SymmetricityinAlgorithmicEquivalence}.
    \end{rneqncase}
    \begin{rneqncase}{\QKRefl}{}
        It is obvious from Lemma \ref{lemma:ReflexivityinAlgorithmicEquivalence}.
    \end{rneqncase}
    \begin{rneqncase}{\QKTrans}{}
        It is obvious from Lemma \ref{lemma:TransitioninAlgorithmicEquivalence}.
    \end{rneqncase}
\end{proof}

From Theorem \ref{theorem:SoundnessOfAlgorithmicTyping} and
\ref{theorem:CompletenessofAlgorithmicTyping}, we can say the original typing
and the algorithmic typing are equivalent.


% !TEX root = ../main.tex

\section{Related Work}
\label{sec:related-work}

\AK{========== REWRTING THIS SECTION NOW ==========}
\AK{========== FROM HERE ==========}

% Theory of Multi-stage Programming

We can say the research of theory of multi-stage programming is start from
binding-time analysis\cite{Nielson1992TwoLevel}. Gl{\"u}ck and J{\o}rgensen
exetended it into multi-level in \cite{GluckJorgensen1995Efficient} and
\cite{GluckJorgensen1996Fast}. The relation between multi-stage programming and
logics are also investigated at the same time. Davies~\cite{Davis1996Temporal}
found the Curry-Howard isomorphism can be extended to relate constructive
temporal logic with binding-time analysis. Davies and
Pfenning~\cite{DaviesPfenning1996Modal} also found the relation between the
intuitionistic modal logic S4 and computation with stages. Taha and Sheard made
MetaML~\cite{TahaSheard1997MetaML} by generalizing above works. MetaML has
explicit annotation for multiple stages and cross-stage persistence, which is
the one of main difficulty of this paper. Moggi
et.al.~\cite{MoggiTahaBenaissaSheard99ESOP} formzalied the type system which
can handle both of open and closed code and Benaissa, et.al. analyzed it from
categorical viewpoint in \cite{BenaissaMoggiTahaSheard1999Logical}. You can
find more precise ealrly history of multi-stage programming in Taha's
dissertion~\cite{Taha1999Multi}. Finally, Taha forumulate the environment
classifiers~\cite{TahaNielsen2003Environment}. It can handle not only open and
closed code but also code of medium state. Calcagno, Moggi and
Taha~\cite{CalcagnoMoggiTaha2004InferenceClassifiers} found that we cannot
infer types of terms without classifier annotations. Therefore, our type
inference algorithm takes stages(classifiers) as its input. Taha's environment
classifiers~\cite{TahaNielsen2003Environment} was extended with mutable
cells~\cite{KiselyovKameyamaSudo2016Refined} and with control
operators~\cite{OishiKameyama2017ControlOperators} by Oishi and Kameyama.
Curry-Howard isomorphism of a calculus with environment classifiers is
investigated by Tsukada and Igarashi~\cite{TsukadaIgarashi2010Logical}. Hanada
and Igarashi extended it with Cross Stage Persistence in
\cite{HanadaIgarashi2014CSP}, which is the basis of our calculus \LMD.

% Applications of Multi-stage programming

We can see runtime-specialization~\cite{ConselNoel1996Runtime},
dynamic compile~\cite{AuslanderPhiliposeChambersEggersBershad1996DynamicCompilation, EnglerHsiehKaashoek1996TickC, PolettoHsiehEnglerKaashoek1999CandTcc, GrantPhiliposeMockChambersEggers1999DyC} or
partial evalutation~\cite{JonesGomardSestoft1993partial} as applications of
multi-stage programming. Although there was no concept of multi-stage
programming in this time, they generated efficient code using runtime
information and execute the code at runtime. Quasiquotation in
Lisp~\cite{Bawden1999Lisp} also can be a kind of multi-stage programming
because it can generate and evaluate code although there is no type system for
safety. As I know, MacroML~\cite{GanzAmrTaha2001MacroML} is the first explicit
application of multi-stage programming. They realize hygenic macro function by
seeing macros as a specialization of multi-stage programming.
MetaOCaml~\cite{CalcagnoTahaHuang2003MetaOCaml, Taha2007Gentle} is implemented
first by Calcagno, Taha and Huang. BER MetaOCaml~\cite{Kisekyov2014BERMetaOCamlBER} is a
reimplementation of MetaOCaml by Kiselyov. You can find many applications in
MetaOCaml to real problems in \cite{Taha2007Gentle, Kiselyov2018Reconcilong}.
Iwaki and Igarashi~\cite{IwakiIgarashi2007CompilerAndVMforML} designed virtual
machine and compiler for a multi-level programming language.

The history of dependent type is start from AUTOMATH
project\cite{DeBruijnNicolaas1970AUTOMATH} which is a language for expressing
mathmatics. The pioneering work by
Martin-L\"{o}f~\cite{MartinLof1973Intuitionistic} affected many latter works. Type
checking of dependent types is discussed in 
\cite{Cardelli1988Typechecking, Coquand1996Typechecking}.
Among many dependent type systems such as
$\lambda^\Pi$~\cite{MeyerReinhold1986Type}, The Calculus of
Constructions~\cite{Coquand1988CoC}, and Edinburgh
LF~\cite{HarperHonsellPlotkin1993Framework}, we base our work on
\LLF~\cite{benjamin2005attapldependent}
(which is quite close to $\lambda^\Pi$ and Edinburgh LF) due to its simplicity.

It is well known that dependent types are useful to express detailed properties
of data structures at the type level such as the size of data
structures~\cite{XiPfenning1998Eliminating,Xi1999Dependent,XiHarper2001DTAL}
and typed abstract syntax
trees~\cite{LeijenMeijer1999DSEC,XiChenChen2003Guarded}.  The vector addition
discussed in Section~\ref{sec:formal} is also such an example.


%%%%%% dependent types for (compile-time) code generation

The use of dependent types for code generation is studied by
Chlipala~\cite{Chlipala2010Ur} and Ebner et
al.~\cite{EbnerUllrichRoeschAvigadMoura2017meta}. They use inductive types to
guarantee the well-formedness of generated code.  Aside from the lack of
quasi-quotation, their systems are for heterogeneous meta-programming and
compile-time code generation and they do not support features for run-time code
generation such as run and CSP, as \LMD{} does.

%%%% index/dependent types for MSP

We discuss earlier attempts at incorporating dependent types into multi-stage
programming.  Pasalic and Taha~\cite{PasalicTahaSheard2002Tagless} designed
\(\lambda_{H\circ}\) by introducing the concept of stage into an existing
dependent type system \(\lambda_H\)~\cite{ShaoSahaTrifonovPapaspyrou2002Certified}.  However,
\(\lambda_{H\circ}\) is equipped with neither run nor CSP.  Forgarty, Pasalic,
Siek and Taha~\cite{FogartyPasalicSiekTaha2007Concoqtion} extended the type system of
MetaOCaml with indexed types.  With this extension, types can be indexed with a
Coq term.  Chen and Xi~\cite{ChenXi2003Meta} introduced code types augmented with
information on types of free variables in code values in order to prevent code
with free variables from being evaluated.  These systems separate the language
of type indices from the term language.  As a result, they do not enjoy
full-spectrum dependent types but are technically simpler because there is no
need to take the stage of types into account.  Brady and
Hammond~\cite{BradyHammond2006Dependently} have discussed a combination of
(full-spectrum) dependently typed programming with staging in the style of
MetaOCaml to implement a staged interpreter, which is statically guaranteed to
generate well-typed code.  However, they focused on concrete programming
examples and there is no theoretical investigation of the programming language
they used. Gratzer et al. introduced modalities in dependent type theory in
\cite{GratzerSterlingBirkedal2019ModalDependent}. It helps us to find the logic which
corresponds to \LMD.

Berger and Tratt~\cite{BergerTratt2015HGRTMP} gave program logic for
\(\text{Mini-ML}^\square_e\)~\cite{DaviesPfenning01JACM}, which would
allow fine-grained reasoning about the behavior of code generators.
However, it cannot manipulate open code which ours can deal with.

\AK{========== UNTIL HERE ==========}

% Practical research on multi-stage programming

MetaOCaml is a programming language with quoting, unquoting, run, and CSP.
Kiselyov~\cite{Kisekyov2018Reconciling} describes many applications of MetaOCaml, including
filtering in signal processing, matrix-vector product, and a DSL compiler.

% The history of multi-stage programming

Theoretical studies on multi-stage programming owe a lot to seminal work by
Davies and Pfenning~\cite{DaviesPfenning01JACM} and
Davies~\cite{davies1996temporal}, who found Curry-Howard correspondence
between multi-stage calculi and modal logic. In particular, Davies'
$\lambda^\circ$~\cite{davies1996temporal} has been a basis for several
multi-stage calculi with quasi-quotation. $\lambda^\circ$ did not have
operators for run and CSP; a few
studies~\cite{BenaissaMoggiTahaSheard1999Logical,MoggiTahaBenaissaSheard99ESOP} enhanced and
improved $\lambda^\circ$ towards the development of a type-safe multi-stage
calculus with quasi-quotation, run, and CSP, which were proposed by Taha and
Sheard as constructs for multi-stage programming~\cite{MetaML}. 
Finally, Taha and Nielsen invented the concept of environment
classifiers~\cite{TahaNielsen2003Environment} and developed a typed calculus
$\lambda^\alpha$, which was equipped with all the features above in a type
sound manner and formed a basis of earlier versions of MetaOCaml. Different
approaches to type-safe multi-stage programming with slightly different
constructs for composing and running code values have been studied by Kim,
Yi, and Calcagno~\cite{KimYi2006PolymorphicModal} and Nanevski and
Pfenning~\cite{NanevskiPfenning2005Staged}.

Later, Tsukada and Igarashi~\cite{TsukadaIgarashi2010Logical} found correspondence
between a variant of \(\lambda^\alpha\) called $\lambda^\TW$
and modal logic and showed that run could be represented as a special
case of application of a transition abstraction ($\Lambda\alpha.M$) to
the empty sequence $\varepsilon$.  Hanada and
Igarashi~\cite{HanadaIgarashi2014CSP} developed \LTP as an extension
$\lambda^\TW$ with CSP.

% The history of dependent types

There is much work on dependent types and most of it is affected by the
pioneering work by Martin-L\"{o}f~\cite{MartinLof1973Intuitionistic}.  Among many
dependent type systems such as $\lambda^\Pi$~\cite{Meyer1986}, The Calculus of
Constructions~\cite{coquand:inria-00076024}, and Edinburgh
LF~\cite{harper1993framework}, we base our work on \LLF~\cite{attapl} (which is
        quite close to $\lambda^\Pi$ and Edinburgh LF) due to its simplicity.
It is well known that dependent types are useful to express detailed properties
of data structures at the type level such as the size of data
structures~\cite{XiPfenning1998Eliminating} and typed abstract syntax
trees~\cite{LeijenMeijer1999DSEC, XiChenChen2003Guarded}. The vector addition
discussed in Section~\ref{sec:formal} is also such an example.

% Applications of dependent types

% Practical applications of dependent types have been also studied. One can use
% dependent types in programming languages such as Idris~\cite{Brady2013Idris}
% or interactive theorem provers such as Coq~\cite{2009Coq} based on
% \cite{coquand:inria-00076024}. In Xi and Pfenning~\cite{Xi98}, they extended
% SML with restricted dependent types and succeeded in reducing the bounds
% checking of arrays. In Xi and Harper~\cite{XiHarpaer2001Dependently}, they design a
% type system for an assembly language and it is useful for speed up. Xi also
% gave dead code elimination and loop unrolling as applications of dependent
% types~\cite{xi1999dependent}.

% Comparison with other works

% Although there are studies on combinations of multi-stage programming and
% other programming features such as mutable cells~\cite{kiselyov2016refined}
% and control operators~\cite{KameyamaKiselyovShan09Shifting,OishiKameyama2017Staging}, a
% combination with dependent types has been little studied.

%%%%%% dependent types for (compile-time) code generation

The use of dependent types for code generation is studied by
Chlipala~\cite{Chlipala2010Ur} and Ebner et
al.~\cite{DBLP:journals/pacmpl/EbnerURAM17}.  They use inductive types to
guarantee the well-formedness of generated code.  Aside from the lack of
quasi-quotation, their systems are for heterogeneous meta-programming and
compile-time code generation and they do not support features for run-time code
generation such as run and CSP, as \LMD{} does.

%%%% index/dependent types for MSP

We discuss earlier attempts at incorporating dependent types into multi-stage
programming.  Pasalic and Taha~\cite{PasalicTahaSheard2002Tagless} designed
\(\lambda_{H\circ}\) by introducing the concept of stage into an existing
dependent type system \(\lambda_H\)~\cite{ShaoSahaTrifonovPapaspyrou2002Certified}.  However,
\(\lambda_{H\circ}\) is equipped with neither run nor CSP.  Forgarty, Pasalic,
Siek and Taha~\cite{FogartyPasalicSiekTaha2007Concoqtion} extended the type system of
MetaOCaml with indexed types.  With this extension, types can be indexed with a
Coq term.  Chen and Xi~\cite{ChenXi2003Meta} introduced code types augmented with
information on types of free variables in code values in order to prevent code
with free variables from being evaluated.  These systems separate the language
of type indices from the term language.  As a result, they do not enjoy
full-spectrum dependent types but are technically simpler because there is no
need to take the stage of types into account.  Brady and
Hammond~\cite{BradyHammond2006Dependently} have discussed a combination of
(full-spectrum) dependently typed programming with staging in the style of
MetaOCaml to implement a staged interpreter, which is statically guaranteed to
generate well-typed code.  However, they focused on concrete programming
examples and there is no theoretical investigation of the programming language
they used. Gratzer et al. introduced modalities in dependent type theory in
\cite{GratzerSterlingBirkedal2019ModalDependent}. It helps us to find the logic which
corresponds to \LMD.

Berger and Tratt~\cite{BergerTratt2015HGRTMP} gave program logic for
\(\text{Mini-ML}^\square_e\)~\cite{DaviesPfenning01JACM}, which would
allow fine-grained reasoning about the behavior of code generators.
However, it cannot manipulate open code which ours can deal with.

% 他の方法としてどのようなものが考えられたか

% Although we define type equivalence of \LMD with composition of equivalence
% rules, there is another candidate to define, which gives reduction on types
% and compare the results of reduction such as~\cite{SorensenUrzyczyn2006Lectures}.
% This method is better than one of \LMD because equivalence rules become
% simple. However, we reject it because it cannot handle CSP flexibly.


% !TEX root = ../main.tex

\section{Conclusion \label{sec:conclusion}}

% What we did

We have proposed a new multi-stage calculus \LMD with dependent types, which make it possible for programmers to express finer-grained properties about the behavior of code values. Because the generated code by multi-stage programming is sometimes specialized in given inputs, it cannot be used with specific inputs. This expansion can assure it is only used with desired inputs.

% Technical Contribution

Technically, CSP and type equivalence (specially tailored for CSP) are keys to expressing dependently typed practical code generators.  We have proved basic properties of \LMD, including preservation, confluence, strong normalization for full reduction, and progress for staged reduction. Furthermore, we design algorithmic typing rules for \LMD, which we need to implement a type inference program and prove it is equivalent to original typing rules.

% Future Works

The main future work is investigating the Curry-Howard isomorphism to find a logic corresponding to \LMD. Tsukada and Igarashi \cite{TsukadaIgarashi2010Logical} found that a kind of modal logic corresponds to stage, and Gratzer et al. \cite{GratzerSterlingBirkedal2019ModalDependent} investigated a dependent type theory with modalities. However, they have not explored the CSP operator, which is the critical point of our research.



% !TEX root = ../main.tex
\section*{Acknowledgments}
First of all, I thank to my supervisor, Prof. Atsushi Igarashi.  Without his
support, I cannot finish my thesis.  He is the most powerful and smart person
who I have met.

Second, I thank to all members in the laboratory. They made my life in my
master cource fruitful and pleasant.

At the last, I am grateful to all restaurants around the laboratory, Cofe
Collection, Matsunosuke, Kitchen Gorilla, and Mikaen. These restaurants
encourage me to proceed my research.



\bibliography{main}

\end{document}
