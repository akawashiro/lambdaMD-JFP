% !TEX root = ../main.tex

\section{Conclusion \label{sec:conclusion}}

% What we did

We have proposed a new multi-stage calculus \LMD with dependent types, which make it possible for programmers to express finer-grained properties about the behavior of code values. Because the generated code by multi-stage programming is sometimes specialized in given inputs, it cannot be used with specific inputs. This expansion can assure it is only used with desired inputs.

% Technical Contribution

Technically, CSP and type equivalence (specially tailored for CSP) are keys to expressing dependently typed practical code generators.  We have proved basic properties of \LMD, including preservation, confluence, strong normalization for full reduction, and progress for staged reduction. Furthermore, we design algorithmic typing rules for \LMD, which we need to implement a type inference program and prove it is equivalent to original typing rules.

% Future Works

The main future work is developing an interactive theorem prover (ITP) based on \LMD. Because types of \LMD can include a code fragment, we can express properties on code using the ITP. For example, it enables us to prove that a given function generates a tail-recursive code fragment or not. To realize it, we must investigate the Curry-Howard isomorphism of \LMD. Tsukada and Igarashi \cite{TsukadaIgarashi2010Logical} found that a kind of modal logic corresponds to stage, and Gratzer et al. \cite{GratzerSterlingBirkedal2019ModalDependent} investigated a dependent type theory with modalities. However, they have not explored the CSP operator, which is the critical point of our research. Furthermore, the ITP needs a special kind of equality to check two terms are equivalent when we see them as code fragments because an efficient code fragment and an inefficient one are equivalent under the equivalence relation of \LMD. Katsuda and Igarashi~\cite{KatsudaIgarashi2020} research a calculus with such equality based on \LMD.
