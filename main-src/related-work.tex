% !TEX root = ../main.tex

\section{Related Work}
\label{sec:related-work}

\AK{========== REWRTING THIS SECTION NOW ==========}
\AK{========== FROM HERE ==========}

% Theory of Multi-stage Programming

We can say the research of theory of multi-stage programming is start from
binding-time analysis\cite{Nielson1992TwoLevel}. Gl{\"u}ck and J{\o}rgensen
exetended it into multi-level in \cite{GluckJorgensen1995Efficient} and
\cite{GluckJorgensen1996Fast}. The relation between multi-stage programming and
logics are also investigated at the same time. Davies~\cite{davis1996temporal}
found the Curry-Howard isomorphism can be extended to relate constructive
temporal logic with binding-time analysis. Davies and
Pfenning~\cite{DaviesPfenning1996Modal} also found the relation between the
intuitionistic modal logic S4 and computation with stages. Taha and Sheard made
MetaML~\cite{TahaSheard1997MetaML} by generalizing above works. MetaML has
explicit annotation for multiple stages and cross-stage persistence, which is
the one of main difficulty of this paper. Moggi
et.al.~\cite{MoggiTahaBenaissaSheard99ESOP} formzalied the type system which
can handle both of open and closed code and Benaissa, et.al. analyzed it from
categorical viewpoint in \cite{BenaissaMoggiTahaSheard1999Logical}. You can
find more precise ealrly history of multi-stage programming in Taha's
dissertion~\cite{taha1999multi}.

\begin{itemize}
    \item \cite{Nielson1992TwoLevel}
\end{itemize}

\AK{========== UNTIL HERE ==========}

% Practical research on multi-stage programming

MetaOCaml is a programming language with quoting, unquoting, run, and CSP.
Kiselyov~\cite{8384206} describes many applications of MetaOCaml, including
filtering in signal processing, matrix-vector product, and a DSL compiler.

% The history of multi-stage programming

Theoretical studies on multi-stage programming owe a lot to seminal work by
Davies and Pfenning~\cite{DaviesPfenning01JACM} and
Davies~\cite{davies1996temporal}, who found Curry-Howard correspondence
between multi-stage calculi and modal logic. In particular, Davies'
$\lambda^\circ$~\cite{davies1996temporal} has been a basis for several
multi-stage calculi with quasi-quotation. $\lambda^\circ$ did not have
operators for run and CSP; a few
studies~\cite{benaissa1999logical,MoggiTahaBenaissaSheard99ESOP} enhanced and
improved $\lambda^\circ$ towards the development of a type-safe multi-stage
calculus with quasi-quotation, run, and CSP, which were proposed by Taha and
Sheard as constructs for multi-stage programming~\cite{MetaML}. 
Finally, Taha and Nielsen invented the concept of environment
classifiers~\cite{taha2003environment} and developed a typed calculus
$\lambda^\alpha$, which was equipped with all the features above in a type
sound manner and formed a basis of earlier versions of MetaOCaml. Different
approaches to type-safe multi-stage programming with slightly different
constructs for composing and running code values have been studied by Kim,
Yi, and Calcagno~\cite{DBLP:conf/popl/KimYC06} and Nanevski and
Pfenning~\cite{DBLP:journals/jfp/NanevskiP05}.

Later, Tsukada and Igarashi~\cite{Tsukada} found correspondence
between a variant of \(\lambda^\alpha\) called $\lambda^\TW$
and modal logic and showed that run could be represented as a special
case of application of a transition abstraction ($\Lambda\alpha.M$) to
the empty sequence $\varepsilon$.  Hanada and
Igarashi~\cite{Hanada2014} developed \LTP as an extension
$\lambda^\TW$ with CSP.

% The history of dependent types

There is much work on dependent types and most of it is affected by
the pioneering work by Martin-L\"{o}f~\cite{martin1973intuitionistic}.
Among many dependent type systems such as
$\lambda^\Pi$~\cite{Meyer1986}, The Calculus of
Constructions~\cite{coquand:inria-00076024}, and Edinburgh
LF~\cite{harper1993framework}, we base our work on \LLF~\cite{attapl}
(which is quite close to $\lambda^\Pi$ and Edinburgh LF) due to its
simplicity.  It is well known that dependent types are useful to
express detailed properties of data structures at the type level such
as the size of data structures~\cite{Xi98} and typed abstract syntax
trees~\cite{DBLP:conf/dsl/LeijenM99,DBLP:conf/popl/XiCC03}.  The
vector addition discussed in Section~\ref{sec:formal} is also such an
example.

% Applications of dependent types

% Practical applications of dependent types have been also studied. One can use
% dependent types in programming languages such as Idris~\cite{brady2013idris}
% or interactive theorem provers such as Coq~\cite{09thecoq} based on
% \cite{coquand:inria-00076024}. In Xi and Pfenning~\cite{Xi98}, they extended
% SML with restricted dependent types and succeeded in reducing the bounds
% checking of arrays. In Xi and Harper~\cite{xi2001dependently}, they design a
% type system for an assembly language and it is useful for speed up. Xi also
% gave dead code elimination and loop unrolling as applications of dependent
% types~\cite{xi1999dependent}.

% Comparison with other works

% Although there are studies on combinations of multi-stage programming and
% other programming features such as mutable cells~\cite{kiselyov2016refined}
% and control operators~\cite{KameyamaKiselyovShan09PEPM,oishi2017staging}, a
% combination with dependent types has been little studied.

%%%%%% dependent types for (compile-time) code generation

The use of dependent types for code generation is studied by
Chlipala~\cite{chlipala2010ur} and Ebner et
al.~\cite{DBLP:journals/pacmpl/EbnerURAM17}.  They use inductive types to
guarantee the well-formedness of generated code.  Aside from the lack of
quasi-quotation, their systems are for heterogeneous meta-programming and
compile-time code generation and they do not support features for run-time code
generation such as run and CSP, as \LMD{} does.

%%%% index/dependent types for MSP

We discuss earlier attempts at incorporating dependent types into multi-stage programming.  Pasalic and Taha~\cite{pasalic2002tagless} designed \(\lambda_{H\circ}\) by introducing the concept of stage into an existing dependent type system \(\lambda_H\)~\cite{zhong2002certified}.  However, \(\lambda_{H\circ}\) is equipped with neither run nor CSP.  Forgarty, Pasalic, Siek and Taha~\cite{fogarty2007concoqtion} extended the type system of MetaOCaml with indexed types.  With this extension, types can be indexed with a Coq term.  Chen and Xi~\cite{chen2003meta} introduced code types augmented with information on types of free variables in code values in order to prevent code with free variables from being evaluated.  These systems separate the language of type indices from the term language.  As a result, they do not enjoy full-spectrum dependent types but are technically simpler because there is no need to take the stage of types into account.  Brady and Hammond~\cite{brady2006dependently} have discussed a combination of (full-spectrum) dependently typed programming with staging in the style of MetaOCaml to implement a staged interpreter, which is statically guaranteed to generate well-typed code.  However, they focused on concrete programming examples and there is no theoretical investigation of the programming language they used. Gratzer et al. introduced modalities in dependent type theory in \cite{gratzer2019modaldependent}. It helps us to find the logic which corresponds to \LMD.

Berger and Tratt~\cite{martin2015HGRTMP} gave program logic for
\(\text{Mini-ML}^\square_e\)~\cite{DaviesPfenning01JACM}, which would
allow fine-grained reasoning about the behavior of code generators.
However, it cannot manipulate open code which ours can deal with.

% 他の方法としてどのようなものが考えられたか

% Although we define type equivalence of \LMD with composition of equivalence
% rules, there is another candidate to define, which gives reduction on types
% and compare the results of reduction such as~\cite{sorensen2006lectures}.
% This method is better than one of \LMD because equivalence rules become
% simple. However, we reject it because it cannot handle CSP flexibly.
